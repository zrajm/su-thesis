%% -*- mode: latex; ispell-dictionary: "svenska" -*-

\section{Att använda mallen}
\label{mallen}


\subsection{Ladda mallen}
\label{ladda}

För att ladda mallen använd \LaTeX{}-kommandot:

\begin{verbatim}
\usepackage{su-thesis}
\end{verbatim}


\subsection{Välja språk}
\label{språk}

\noindent Språket på uppsatsen ställer man in med hjälp av paketet
\texttt{babel}. När du byter språk så kommer byts fram- och baksida ut mot
information på det relevanta språket (men resten av uppsatsen får du förstås
översätta själv). :)

\begin{verbatim}
\usepackage[swedish]{babel} % Swedish
%% \usepackage[UKenglish]{babel} % British English
%% \usepackage[USenglish]{babel} % American English
\end{verbatim}

\noindent\textbf{Notera:} När du bytt språk behöver du radera alla tempfiler
och kompilera om. Om du använder \texttt{latexmk} är detta enkelt.

\begin{verbatim}
latexmk -c su-thesis.tex  % radera alla tempfiler
latexmk su-thesis.tex     % kompilera om
\end{verbatim}


\subsection{Färger}
\label{färg}

Uppsatsmallen ger dig tillgång till följande färger kommer från Stockholms
universitets grafiska profil. Använd kommandot \texcommand[NAMN]{color} med
färgens namn för att byta färg på texten (tex används
\texcommand[sublue]{color} för att sätta färg på uppsatsens fram- och
baksida).

\medskip

\begin{center}
  \begin{tabular}{ccc}
    \toprule
    \thead{namn} & \thead{färg} & \thead{hexkod} \\
    \midrule
    sublue  &  \color{sublue}\rule{3em}{1.5em} & \texttt{\#002F5F} \\
    suolive & \color{suolive}\rule{3em}{1.5em} & \texttt{\#A3A86B} \\
    susky   &   \color{susky}\rule{3em}{1.5em} & \texttt{\#ACDEE6} \\
    suwater & \color{suwater}\rule{3em}{1.5em} & \texttt{\#9BB2CE} \\
    sufire  &  \color{sufire}\rule{3em}{1.5em} & \texttt{\#D95E00} \\
    \bottomrule
  \end{tabular}
\end{center}

\medskip

Färgerna är definierade med \LaTeX{}-paketet "\texttt{xcolor}", så för mer
information om hur du kan använda dem, se dokumentationen du hittar här:
\url{https://ctan.org/pkg/xcolor}.


\subsection{Mallens variabler}
\label{variabler}

Mallen har ett antal variabler som man sätter med \verb|\suset[SPRÅK]{NAMN}|
(för tillfället kan bara \texttt{swedish} och \texttt{english} användas som
språk).

\textbf{Notera:} Du behöver ladda mallen innan du kan sätta variablerna.

\begin{verbatim}
%% Always used
\suset{author}{Förnamn Efternamn}
\suset{supervisor}{Förnamn1 Efternamn1[, Förnamn2 Efternamn2]}
\suset[swedish]{title}{Titel på svenska}
\suset[english]{title}{Title in English}
\suset[swedish]{abstract}{Sammanfattning på svenska}
\suset[english]{abstract}{Abstract in English}

%% Swedish metadata (only used if language is Swedish)
\suset[swedish]{subtitle}{Svensk undertitel}
\suset[swedish]{keywords}{nyckelord1, nyckelord2, nyckelord3, \ldots}
\suset[swedish]{department}{Institutionen för lingvistik}
\suset[swedish]{thesistype}{Examensarbete 15 hp}
\suset[swedish]{course}{Lingvistik -- kandidatkurs, LIN600}
\suset[swedish]{program}{Kandidatprogrammet i lingvistik 180 hp}
\suset[swedish]{semester}{Vårterminen 2020}

%% English metadata (only used if language is English)
\suset[english]{subtitle}{English Subtitle}
\suset[english]{keywords}{keyword1, keyword2, keyword3, \ldots}
\suset[english]{department}{Department of Linguistics}
\suset[english]{thesistype}{Bachelor's Thesis 15 ECTS credits}
\suset[english]{course}{Linguistics -- Bachelor's Course, LIN600}
\suset[english]{program}{Bachelor's Programme in Linguistics 180 ECTS credits}
\suset[english]{semester}{Spring semester 2020}
\end{verbatim}


\subsection{Fonetisk transkription (IPA)}
\label{ipa}

Den version av Times New Roman som kommer med uppsatsmallen\footnote{Times New
  Roman har haft stöd för IPA-symboler sedan version 5.22 \citep{ipa} som kom
  med Windows 7 \citep{win-7-fonts}. I uppsatsmallen är Times New Roman version
  7.00 och Verdana 5.22 inkluderade (från Windows 10).} har stöd för
IPA-symboler (International Phonetic Alphabet). Detta fungerar med både fet och
kursiv stil fungerar, men dessvärre finns inte IPA-symbolerna med i Verdana, så
det går bara att få fonetiska symboler med serifftypsnitt. Tabell
\ref{tab-ipa} är ett exempel på hur det kan se ut.

Om du saknar en inmatningsmetod för att skriva IPA-symboler så kan du använda
följande hemsida och därefter klippa/klistra in resultatet i din uppsats:
\url{https://ipa.typeit.org/full/}. (Det går förstås lika bra att använda Word,
eller andra program och klistra in resultatet därifrån i ditt dokument.)

\begin{table}
  \caption{IPA-exempel, svenska konsonanter (från \citealp[140]{engstrand-1999},
    egen översättning).}
  \label{tab-ipa}
  \vspace{.5em}
  %%
%% This table uses 'tabularx', 'array' and 'makecell' packages.
%%
%% * 'Tables in LATEX2ε: Packages and Methods' (docs on tables etc.)
%%   https://www.tug.org/pracjourn/2007-1/mori/mori.pdf
%%
%% * 'The tabularx package∗'
%%   http://mirrors.ibiblio.org/CTAN/macros/latex/required/tools/tabularx.pdf
%%
%%
%% Random Observations
%% ===================
%%
%%       \setlength{\extrarowheight}{40pt}%   % array:
%%       \setlength{\tabcolsep}{<length>}     % tabular: ½ of width between cols
%%       \setlength{\arraycolsep}{<length>}   % array:   ½ of width between cols
%%       %%\extracolsep{\fill}
%%
%%   Vertical padding is possible in a global way using @Herbert's answer. That
%%   is, to redefine the array stretch factor <factor> using
%%
%%       \renewcommand\arraystretch{.01}%      % (array)
%%
%%   \baselineskip -- Specifies minimum space between baselines successive
%%   lines in a paragraph. Is reset, for example, by font changes. The value in
%%   effect at end of a paragraph, is used for whole paragraph.
%%
%%       \setlength{\baselineskip}{0pt}%      % minimum line height (reset by font commands)
%%
%%   \baselinestretch is used to multiply \baselineskip. (Default: 1.0). Use
%%   this to change line height in a document, since its not reset by other
%%   commands.
%%
%%       \renewcommand{\baselinestretch}{0}%   % multiplier of line height (default: 1)
%%       \arraybackslash%                      % '\\' inside table
%%
%%
%% Column Specifications
%% =====================
%% These column specifications are used by 'tabularx'. Some of them are
%% imported from the 'array' package, and some from the (non-x) 'tabular'.
%%
%%     @{text}     Replace padding between this & previous column with <text>
%%     >{code}     (array) insert <code> at beginning of cell
%%     l           left aligned
%%     r           right aligned
%%     c           horizontally centered
%%     p{width}    justified
%%     m{width}    (array) vertically centered
%%     b{width}    (array) bottom align
%%     <{code}     (array) insert <code> at end of cell
%%     @{text}     Replace padding between this & next column with <text>
%%
%% Relative Column Widths ('tabularx')
%% ===================================
%% If there are two columns, then then all <\hsize>s added together should be
%% 2, but if you want differing widths you could use something like this (which
%% will cause the first column to be three times wider than the second):
%%
%%    \begin{tabularx}{\linewidth}{|%
%%      >{\hsize=1.5\hsize\linewidth=\hsize}c|%
%%      >{\hsize=0.5\hsize\linewidth=\hsize}c|}
%%
\centering%
{
  \large%
  \newcommand{\na}{\makecell{}}               % empty cell content
  \newcommand{\td}[1]{\makecell{#1}}
  \renewcommand{\theadfont}{\normalsize}      % small font in header
  \renewcommand\theadalign{cc}
  \renewcommand\cellalign{cc}
  \renewcommand\theadset{                     % (makecell)
    \renewcommand\arraystretch{.6}%           % (array) line height
    %\renewcommand\theadalign{c}
  }%
  \renewcommand{\tabularxcolumn}[1]{>{}m{#1}}
  \newcolumntype{Z}{%                       % phoneme column
    >{\centering\arraybackslash}X%
  }%
  \newcolumntype{P}{%                       % Placeholder column
    @{}%                                    %   suppress left margin
    c%                                      %   horizontally centered text
    @{}%                                    %   suppress right margin
  }%
  %% Left side table headers.
  %%
  %% Leading & trailing space should be of same width here. But something is
  %% adding approx .2em space on the left side, so that's from the left hand
  %% side subtracted below. (Specifying an empty @{} will suppress the
  %% default space between columns.)
  %%
  \newcolumntype{H}{%             % Header
    @{\hspace{.3em}}%             %   left margin
    c%                            %   horizontally centered text
    @{\hspace{.5em}}%             %   right margin
  }%
  \begin{tabularx}{\linewidth}{P|H|ZZ|ZZ|ZZ|ZZ|ZZ|ZZ|ZZ|}
    \hline%---------------------------------------------------------------------
    \makecell{\rule{0pt}{1.5em}} &
                                                & % header = 3 columns wide
    \multicolumn{2}{c|}{\thead{Bilabial}}       &
    \multicolumn{2}{c|}{\thead{Labio-\\dental}} &
    \multicolumn{2}{c|}{\thead{Dental}}         &
    \multicolumn{2}{c|}{\thead{Alveolar}}       &
    \multicolumn{2}{c|}{\thead{Palatal}}        &
    \multicolumn{2}{c|}{\thead{Velar}}          &
    \multicolumn{2}{c|}{\thead{Glottal}} \\
    \hline%---------------------------------------------------------------------
    \makecell{\rule{0pt}{1.5em}} &
    \thead{Klusil} &
    \td{p} & \td{b} &  % bilabial
    \na    & \na    &  % labiodental
    \td{t} & \td{d} &  % dental
    \na    & \na    &  % alveolar
    \na    & \na    &  % palatal
    \td{k} & \td{ɡ} &  % velar
    \na    & \na    \\ % glottal
    \hline%---------------------------------------------------------------------
    \makecell{\rule{0pt}{1.5em}} &
    \thead{Nasal} &
    \na & \td{m} &  % bilabial
    \na & \na    &  % labiodental
    \na & \td{n} &  % dental
    \na & \na    &  % alveolar
    \na & \na    &  % palatal
    \na & \td{ŋ} &  % velar
    \na & \na    \\ % glottal
    \hline%---------------------------------------------------------------------
    \makecell{\rule{0pt}{1.5em}} &
    \thead{Frikativa} &
    \na    & \na    &  % bilabial
    \td{f} & \td{v} &  % labiodental
    \td{s} & \na    &  % dental
    \na    & \na    &  % alveolar
    \na    & \td{ʝ} &  % palatal
    \na    & \na    &  % velar
    \td{h} & \na    \\ % glottal
    \hline%---------------------------------------------------------------------
    \makecell{\rule{0pt}{1.5em}} &
    \thead{Approximant} &
    \na & \na    &  % bilabial
    \na & \na    &  % labiodental
    \na & \na    &  % dental
    \na & \td{ɹ} &  % alveolar
    \na & \na    &  % palatal
    \na & \na    &  % velar
    \na & \na    \\ % glottal
    \hline%---------------------------------------------------------------------
    \makecell{\rule{0pt}{1.5em}} &
    \thead{Lateral\\approximant} &
    \na & \na    &  % bilabial
    \na & \na    &  % labiodental
    \na & \td{l} &  % dental
    \na & \na    &  % alveolar
    \na & \na    &  % palatal
    \na & \na    &  % velar
    \na & \na    \\ % glottal
    \hline%---------------------------------------------------------------------
  \end{tabularx}

  \medskip

  ɧ {\normalsize \hspace{.25em} Tonlös dorso-palatal/velar frikativa}
  \hspace{.8em}
  ɕ {\normalsize \hspace{.25em} Tonlös alveolar-palatal frikativa}
}

%% [eof]

  \vspace{1em}
\end{table}


\subsection{Teckenspråkstranskription}
\label{teckenspråk}

Med uppsatsmallen kommer sansserifftypsnittet FreeSans-SWL ("SWL" är ISO-koden
för svenskt teckenspråk) som innehåller de teckentranskriptionssymboler som
används för svenskt teckenspråk. (Symbolerna används inte utanför Sverige och
det finns dessvärre ingen version av Verdana eller Times New Roman som
inkluderar dem.)

Det saknas stöd för teckentranskriptionssymbolerna i de flesta
textredigeringsprogram, så med största sannolikhet kommer du inte att kunna se
dem när du jobbar med texten. (Om du vill försöka kan du kopiera typsnittet
från "\texttt{fonts/freesans-swl.ttf}" och installera det på din dator och
därefter testa att välja det som typsnitt i ditt textredigeringsprogram.)

Det enklaste sättet att skriva symbolerna är att använda följande webbsida, och
därefter klistra in resultatet därifrån in i ditt \LaTeX{}-dokument:
\href{https://zrajm.github.io/teckentranskription/}{\url{zrajm.github.io/teckentranskription/}}

Det finns två \LaTeX{} kommandon för att använda det: \texcommand{swl} och
\verb|\swlfamily|. \texcommand{swl} är i det flesta situationer det mer
praktiska av de två. Ange den text du vill visa med FreeSans-SWL som argument
(vanligtvis är detta transkriptionen, men FreeSans-SWL är ett Unicode-typsnitt
och har support för de flesta språk).

\medskip

\hspace{1em}\texcommand[????????????????]{swl} → \swl{􌥃􌥔􌥘􌥃􌤵􌤷􌥧􌥡􌥼􌥲􌦊􌥱􌦈􌥼􌤟􌥣}

\medskip

\verb|\swlfamily| byter typsnitt till FreeSans-SWL och passar bättre till
längre texter (jag har aldrig sätt en längre text skriven på detta sätt). För
att byta tillbaka till originaltypsnittet kan du antingen använda
\verb|{\swlfamily …}|, eller använda \verb|\rmfamily| för att byta tillbaka
till Times New Roman efteråt, eller använda \verb|\sffamily| för att byta till
Verdana.

\begin{verbatim}
\swlfamily
Teckenspråk heter ???????????????? på teckenspråk.
\rmfamily
\end{verbatim}

\swlfamily
\hspace{1em}→ Teckenspråk heter 􌥃􌥔􌥘􌥃􌤵􌤷􌥧􌥡􌥼􌥲􌦊􌥱􌦈􌥼􌤟􌥣 på teckenspråk.
\rmfamily


\subsection{Ta bort inledande/avslutande sidor}
\label{genererade}

Mallen använder interna kommandon för att genera inledande och avslutande
sidor. För att ta bort en eller flera av dessa kan man definiera om de
relevanta kommandona så att det inte generar någon effekt. Detta görs med hjälp
av \LaTeX{}-kommandot \verb|\renewcommand|. Följande kommandon kan användas:

\begin{verbatim}
\usepackage{su-thesis}
...
\renewcommand{\sumakefrontpage}{}     % ta bort framsida
\renewcommand{\sumakeabstractpage}{}  % ta bort sammanfattningssida
\renewcommand{\sutableofcontents}{}   % ta bort innehållsförteckning
\renewcommand{\sumakebackpage}{}      % ta bort baksida
\end{verbatim}

\noindent\textbf{Notera:} Raden med \verb|\renewcommand| måste utföras efter
\verb|\usepackage{su-thesis}|.

%% [eof]

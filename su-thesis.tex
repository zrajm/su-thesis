%% -*- mode:latex; ispell-dictionary: "svenska" -*-
%% 'hidelinks' removes colored borders around links etc.
\documentclass[hidelinks,11pt]{article}

%% Nowidow's documentation:
%% http://ftp.acc.umu.se/mirror/CTAN/macros/latex/contrib/nowidow/nowidow.pdf
\usepackage[all]{nowidow} % Tries to remove widows

%% Microtype's documentation:
%% http://ftp.acc.umu.se/mirror/CTAN/macros/latex/contrib/microtype/microtype.pdf
\usepackage{microtype} % Improves typography, load after fontpackage is selected

\usepackage[swedish]{babel} % Swedish
%% \usepackage[UKenglish]{babel} % British English
%% \usepackage[USenglish]{babel} % American English

\usepackage{makecell} % https://ctan.org/pkg/makecell -- \makecell{} \thead{} \multirowthead etc
\renewcommand{\theadfont}{\normalsize\sffamily\bfseries}
\renewcommand{\theadalign}{tl}

\newcommand{\texcommand}[2][…]{\texttt{\textbackslash#2\{#1\}}}

\usepackage{su-thesis}

%% Always used
\suset{author}{Zrajm}
\suset{supervisor}{Ingen}
\suset[swedish]{title}{Mall för uppsats vid Stockholms~universitet}
\suset[swedish]{subtitle}{För \LaTeX{} och Overleaf}
\suset[english]{title}{Thesis Template for Stockholm~University}
\suset[english]{subtitle}{For \LaTeX{} and Overleaf}
\suset[swedish]{keywords}{
  Stockholms universitet, uppsatsmall, nybörjarguide till \LaTeX{},
  kandidatuppsats, masteruppsats, uppsatsskrivande
}
\suset[english]{keywords}{
  Stockholm university, thesis template, beginner's guide to \LaTeX{},
  bachelor's thesis, master's thesis, thesis writing
}
\suset[swedish]{abstract}{
  Detta är en introduktion till hur man använder Stockholms universitets
  uppsatsmall för \LaTeX{}. Här går vi igenom fyra saker: (a) Hur Stockholms
  universitets uppsatsmall ser ut. (b) Vilka möjligheter du, som student vid
  Stockholms universitet, har att använda \LaTeX{}. (c) Hur denna
  \LaTeX{}-version av uppsatsmallen används. Och slutligen (d) går vi i grova
  drag igenom strukturen på hur en uppsats på kandidat- eller masternivå brukar
  se ut på Institutionen för lingvistik. Denna text tjänar också som ett
  exempel på en text skriven i \LaTeX{} och är tänkt att kunna användas som
  utgångspunkt vid uppsatsskrivande.
}
\suset[english]{abstract}{
  This is an introduction to Stockholm university's thesis template for
  \LaTeX{}. In this guide we will cover four things: (a) What thesis template
  of Stockholm university looks like. (b) What choices you, as a student at
  Stockholm university, have for running \LaTeX{}. (c) How to use this \LaTeX{}
  version of the template. And, finally, we also cover roughly (d) what the
  expected structure of a bachelor's or master's thesis look like at the
  Department of Linguistics. This text also serve as an example of a text
  written in \LaTeX{} and is intended as a possible starting point when writing
  a thesis.
}

%% Swedish metadata (only used if language is Swedish)
\suset[swedish]{department}{Institutionen för lingvistik} % e.g. ''
\suset[swedish]{thesistype}{Examensarbete X hp/Degree X HE credits} % e.g. 'Examensarbete 15 hp'
\suset[swedish]{course}{Examensämne} % e.g. 'Lingvistik -- kandidatkurs, LIN600'
\suset[swedish]{program}{Kandidatprogrammet i lingvistik 180 hp}
\suset[swedish]{semester}{Vårterminen 2020}

%% English metadata (only used if language is English)
\suset[english]{department}{Department of Linguistics}
\suset[english]{thesistype}{Bachelor's Thesis 15 ECTS credits}
\suset[english]{course}{Thesis Subject} % e.g. 'Linguistics -- Bachelor's Course, LIN600'
\suset[english]{program}{Bachelor's Programme in Linguistics 180 ECTS credits}
\suset[english]{semester}{Spring semester 2020}

%% Hyphenation patterns
\hyphenation{Fo-ne-tik-lab-bet}

%% Remove margin around lists
\usepackage{enumitem}
\setlist{itemsep=0pt}

%% For XeLaTeX logo
\usepackage{metalogo} % https://ctan.org/pkg/metalogo

%% Suppress abstract page.
%\renewcommand{\sumakeabstractpage}{}

\usepackage{tabularx}
\usepackage{longtable}
\usepackage{multicol}

%% Indent the verbatim environment one em.
%% [https://tex.stackexchange.com/a/148448/185988]
\usepackage{verbatim}
\newlength\myverbindent
\setlength\myverbindent{1em} % change this to change indentation
\makeatletter
\def\verbatim@processline{%
  \hspace{\myverbindent}\the\verbatim@line\par}
\makeatother

%%%%%%%%%%%%%%%%%%%%%%%%%%%%%%%%%%%%%%%%%%%%%%%%%%%%%%%%%%%%%%%%%%%%%%%%%%%%%%%%
\begin{document}

%% \include{} automatically does leading & trailing \clearpage.
%% -*- mode: latex; ispell-dictionary: "svenska" -*-

\section{Stockholms universitets uppsatsmall}
\label{utseende}

Denna uppsatsmall bygger delvis på Stockholms universitets studentuppsatsmall
för Word, delvis på andra stilrekommendationer från Stockholms universitet som
jag kunnat hitta. Jag har också undersökt och jämfört ett tiotal
kandidatuppsatser skrivna vid institutionen för lingvistik och därifrån hämtat
inspiration till sådant som jag inte hittat beskrivet.

\medskip

\begin{itemize}
\raggedright
\item Stockholms universitetsbiblioteks studentuppsatsmall (finns bara till
  Word):\\
  \url{https://su.se/biblioteket/guider/guider/studentuppsatsmall}

\item Stockholms universitetsbiblioteks "Mallar och visuell identitet för
  avhandling" \\
  (Word-mallar för olika typer av avhandlingar): \\
  \url{http://su.se/biblioteket/avhandlingsmallar}

\item Stockholms universitetsbiblioteks "Dokumentmall i Word för
  doktorsavhandlingar" (PDF): \\
  \url{https://su.se/polopoly\_fs/1.264229.1576661286!/menu/standard/
    file/Instruktioner%20SU\_Wordmall\_20180509.pdf}

\item Institutionen för svenska och flerspråkighets "Uppsatsmall och
  Word-tips": \\
  \url{https://su.se/svefler/publikationer/uppsatsarkiv/
    uppsatsmall-och-word-tips}

\item Stockholms universitets grafiska manual (allmän information om hur
  grafisk profil, ingenting specifikt om utseende på uppsatser/avhandlingar):
  \\
  \url{https://su.se/medarbetare/kommunikation/grafisk-manual}
\end{itemize}

%% \subsection{Typsnitt}

%% Brödtexten är satt i Times New Roman, medan titel, rubrik, underrubriker etc.
%% är satta i Verdana. På fram- och baksida är all text och universitetets logotyp
%% satta i Stockholms universitetets blåa färg, på övriga sidor är all text svart.

%% %% Brödtext (12 uppsatser):
%% %%  Blankrad: 11 / Indrag: 1 (Bark 2018)
%% %%  Vänsterjusterad: 7, Spärrad: 5 (Bäckström 2015, Lyxell 2016, Petersdotter
%% %%                                 2018, Bark 2018, Klintberg 2018)

%% \subsubsection{Times New Roman}

%% Brödtexten används för den huvudsakliga texten i uppsatsen. Den används också
%% för texten i styckena för sammanfattning/abstract, och nyckelord/keywords. Den
%% används däremot \emph{inte} för innehållsförteckning eller i textrutorna på
%% fram- och baksida.

%% %\textbf{Styckeindrag} -- Inget.

\subsection{Textstilar}

För att \emph{kursivera text} använd \texcommand{emph}. För \textbf{fetstil}
använd \texcommand{textbf}. Om du vill skriva med \texttt{skriv\-maskins\-stil}
(vilket ofta används för variabelnamn eller kod i text om programmering) använd
\texcommand{texttt}. Du kan också \underline{stryka under text} med
\texcommand{underline}. Det finns många andra \LaTeX{}-stilkommandon du också
kan använda men dessa är de vanligaste.

\LaTeX{} är oftast rättså bra på att luska ut hur ord ska avstavas men ibland
kan det gå snett och ett ord kan sticka långt ut i marginalen. För att fixa
detta när det händer det kan du antingen skriva \verb|\-| inuti ordet där du
vill tillåta avstavning (dvs \verb|Fo\-ne\-tik\-lab\-bet|). Alternativt så kan
du lägga till ett skapa en lista i början av ditt dokument (den måste komma
före \verb|\begin{document}|) med alla avstavningsmönster du behöver.

\begin{verbatim}
\hyphenation{Fo-ne-tik-lab-bet}
\hyphenation{kort-films-festi-vals-med-ar-betar-fest}
\end{verbatim}

En fördel med att lägga till avstavningsmönster i början av dokumentet är att
om ett ord förekommer flera gånger så behöver du bara beskriva avstavningen på
ett ställe. En annan fördel är att du kan söka på ordet i din text och inte
måste komma ihåg att just det ordet var avstavat.

För att skriva webbadresser i din text använd kommandot \texcommand{url}. Det
ser ut såhär i text \url{http://www.exempel.com/}, skapar en länk i texten och
hanterar eventuella radbrytningar på ett bra sätt (utan att stoppa in
bindestreck som förändrar vart adressen pekar).


\subsection{Rubriker och brödtext}

Sidmarginalerna är 2,5~cm på alla fyra sidor, utom upptill där den är 2,725~cm.
Rubriker är vänsterjusterade och satta med Verdana, medan brödtext har rak
högermarginal och är satt med Times New Roman. Stycken är separerade med ett
vertikalt mellanrum utan styckeindrag.

\medskip

\begin{tabular}{p{.33\textwidth} p{.67\textwidth}}
  \toprule
  \thead{Typ} &
  \thead{Beskrivning} \\
  \midrule
%%% Wordmall: 'Heading 1' (sans)
  \thead{
    \Large\sffamily\mdseries
    1~Rubrik
  } &
  \makecell[{{p\linewidth}}]{
    \texcommand{section} \\
    Numrerad kapitelrubrik. Verdana 26~punkter. Vänsterjusterad.
    % I Libreoffice: (Line spacing "at least" 15~punkter.)
    Avstånd ovanför 36~punkter, avstånd nedanför 24~punkter.
  } \\
  \midrule
%%% Wordmall: 'Heading 2' (sans) -- \subsection{}
  \thead{
    \large\sffamily\mdseries
    1.1~Underrubrik
  } &
  \makecell[{{p\linewidth}}]{
    \texcommand{subsection} \\
    Numrerad underrubrik. Verdana 16~punkter fetstil. Vänsterjusterad.
    % I Libreoffice: (Line spacing "at least" 15~punkter.)
    Avstånd ovanför 24~punkter, avstånd nedanför 6~punkter.
  } \\
  \midrule
%%% Wordmall: 'Heading 3' (sans) -- \subsubsection{}
  \thead{
    \normalsize\sffamily\bfseries
    1.1.1~Under-underrubrik
  } &
  \makecell[{{p\linewidth}}]{
    \texcommand{subsubsection} \\
    Numrerad under-underrubrik. Verdana 11~punkter fetstil. Vänsterjusterad.
    % I Libreoffice: (Line spacing "at least" 15~punkter.)
    Avstånd ovanför 12~punkter, avstånd nedanför 6~punkter.
  } \\
  \midrule
%%% Wordmall: 'Default Style'
  %% Set with:
  %%   \documentclass[11pt]{article} % set base font size
  %%   \setlength{\parindent}{0cm}   % remove 1st line of paragraph indentation
  %%   \setlength{\parskip}{6pt}     % add vertical space between paragraphs
  \thead{
    \normalsize\rmfamily\mdseries
    Brödtext
  } &
  \makecell[{{p\linewidth}}]{
    Times New Roman 11~punkter. Vänsterjusterad.
    % I Libreoffice: (Line spacing: Single.)
    Avstånd ovanför 0~punkter, avstånd nedanför 6~punkter.
  } \\
  \bottomrule
%% %%% Wordmall: 'Header'
%%   Sidhuvud &
%%   \hl{FIXME}
%%   \\
%%   \midrule
%% %%% Wordmall: 'Footer'
%%   Sidfot &
%%   \hl{FIXME}
%%   \\
%%   \midrule
\end{tabular}


\subsection{Fram- och baksida}

Texten på fram- och baksidor är helt och hållet satt med Verdana (dvs
sansseriff) och har Stockholms universitets blå färg (på både text och
logotyp).

\medskip

\begin{tabular}{p{.33\textwidth} p{.67\textwidth}}
  \toprule
  \thead{Typ} &
  \thead{Beskrivning} \\
  \midrule
%%% Wordmall: 'Uppsatstitel' (sans, blå)
  \thead{
    \color{sublue}\LARGE\sffamily\mdseries
    Titel
  } &
  \makecell[{{p\linewidth}}]{
    \texcommand{suset[LANG]\{title\}} \\
    Uppsatstiteln på framsidan. Verdana 28~punkter rak. Vänsterjusterad.
    % I Libreoffice: (Line spacing "at least" 15~punkter.)
  } \\
  \midrule
%%% Wordmall: 'Undertitel' (sans, blå)
  \thead{
    \color{sublue}\subtitlesize\sffamily\mdseries
    Undertitel
  } &
  \makecell[{{p\linewidth}}]{
    \texcommand{suset[LANG]\{title\}} \\
    Undertitel och författarnamn på framsidan. Verdana 14~punkter rak.
    Vänsterjusterad.
    % I Lireoffice: (Line spacing "at least" 24~punkter.) Avstånd ovanför
    % 6~punkter, avstånd nedanför 28~punkter.
  } \\
  \midrule
%%% Wordmall: 'Textruta' (sans, blå)
  \thead{
    \color{sublue}\small\sffamily\mdseries
    Textruta
  } &
  \makecell[{{p\linewidth}}]{
        \texcommand{suget[LANG]\{department\}} \\
        \texcommand{suget[LANG]\{thesistype\}} \\
        \texcommand{suget[LANG]\{course\}} \\
        \texcommand{suget[LANG]\{program\}} \\
        \texcommand{suget[LANG]\{semester\}} \\
        \texcommand{suget\{supervisor\}} \\
    Textruta med tryckinformation längst ned till vänster på fram- och baksida.
    Verdana 8~punkter rak. Vänsterjusterad.
    % I Libreoffice: (Line spacing "fixed" 14~punkter.) Avstånd ovanför
    % 0~punkter, avstånd nedanför 0~punkter.
  } \\
  \bottomrule
\end{tabular}


\subsection{Sammanfattningssida}

Detta är sidan efter framsidan. Den innehåller uppsats titel, undertitel och
författarnamn följt av en sammanfattningen och nyckelorden, och därefter den
engelska sammanfattningen (eller 'abstract') och nyckelorden.

Om språket är svenska (specificerat med \verb|\usepackage[swedish]{babel}|) så
visas den svenska titeln, och den svenska sammanfattningen före det engelska
abstractet, i annat fall visas den engelska titeln och det engelska abstractet
före den svenska sammanfattningen.

\medskip

\begin{tabular}{p{.33\textwidth} p{.67\textwidth}}
  \toprule
  \thead{Typ} &
  \thead{Beskrivning} \\
  \midrule
%%% Wordmall: 'Uppsatstitel sidan2' (sans) -- \section*{}
  \thead{
    \Large\sffamily\mdseries
    Titel
  } &
  \makecell[{{p\linewidth}}]{
    %% Verdana 26~punkter. Vänsterjusterad. (Line spacing "at least"
    %% 15~punkter.) Avstånd ovanför 36~punkter, avstånd nedanför 24~punkter \\
    Uppsatstiteln (samma som '1 Rubrik').
  } \\
  \midrule
%%% Wordmall: 'Undertitel sidan2' (sans)
  \thead{
    \normalsize\sffamily\bfseries
    Undertitel
  } &
  \makecell[{{p\linewidth}}]{
    Används på sammanfattningssidan till undertitel och författarnamn på.
    Verdana 11~punkter, fetstil. Vänsterjusterad.
    % I Libreoffice: (Line spacing "at least" 15~punkter.)
    Avstånd ovanför 36~punkter, avstånd nedanför 18~punkter (samma som
    under-underrubrik men med större avstånd ovanför och under).
  } \\
  \midrule
%%% Wordmall: 'Sammanfattning' (sans) -- \section*{}
  \thead{
    \Large\sffamily\mdseries
    Rubrik
  } &
  \makecell[{{p\linewidth}}]{
    Onumrerad rubrik till sammanfattning/abstract
    %% Verdana 26~punkter. Vänsterjusterad. (Line spacing "at least"
    %% 15~punkter.) Avstånd ovanför 36~punkter, avstånd nedanför 24~punkter
    (samma som '1 Rubrik').
  } \\
  \midrule
%%% Wordmall: 'Nyckelord' (sans) samma som 'Heading 3'?)
  \thead{
    \normalsize\sffamily\textbf
    Nyckelordsrubrik
  } &
  \makecell[{{p\linewidth}}]{
    Onumrerad under-underrubrik. Verdana 11~punkter fetstil. Vänsterjusterad.
    % I Libreoffice (Line spacing "at least" 15~punkter.)
    Avstånd ovanför 24~punkter, avstånd nedanför 6~punkter (samma som
    under-underrubrik, men med dubbelt så stort avstånd ovanför).
  } \\
  \bottomrule
\end{tabular}


%%% Wordmall: 'Uppsatstitel' (sans, blå)
%% \textbf{Titel} -- Används bara på framsidan, till uppsatstiteln. Verdana
%% 18~punkter. Vänsterjusterad. (Line spacing "at least" 15~punkter.) Färg: blå.

%% %%% Wordmall: 'Undertitel' (sans, blå)
%% \textbf{Undertitel} -- Används bara på framsidan till undertitel och
%% författarnamn. Verdana 14~punkter, rak. Vänsterjusterad. (Line spacing "at
%% least" 24~punkter.) Avstånd ovanför 6~punkter, avstånd nedanför 28~punkter.
%% Färg: blå.

%% %%% Wordmall: 'Textruta' (sans, blå)
%% \textbf{Textruta} -- Används bara på fram- och baksida till tryckinformation
%% (textrutan längst ned till vänster). Verdana 8~punkter, rak.
%% Vänsterjusterad. (Line spacing "fixed" 14~punkter.) Avstånd ovanför 0~punkter,
%% avstånd nedanför 0~punkter. Färg: blå.

%% %%% Wordmall: 'Uppsatstitel sidan2' (sans)
%% \textbf{Titel sida 2} -- Används bara på sammanfattningssidan. Verdana
%% 26~punkter. Vänsterjusterad. (Line spacing "at least" 15~punkter.) Avstånd
%% ovanför 36~punkter, avstånd nedanför 24~punkter.

%% %%% Wordmall: 'Undertitel sidan2' (sans)
%% \textbf{Undertitel sida 2} -- Används bara på sammanfattningssidan till
%% undertitel och författarnamn på framsidan. Verdana 11~punkter, fetstil.
%% Vänsterjusterad. (Line spacing "at least" 15~punkter.) Avstånd ovanför
%% 36~punkter, avstånd nedanför 18~punkter.

%% %%% Wordmall: 'Heading 1' (sans)
%% \textbf{Rubrik (rubrik 1)} -- Numrerad kapitelrubrik.

%% %%% Wordmall: 'Heading 2' (sans)
%% \textbf{Underrubrik (rubrik 2)} -- Numrerad underrubrik.

%% %%% Wordmall: 'Heading 3' (sans)
%% \textbf{Under-underrubrik (rubrik 2)} -- Numrerad under-underrubrik.

%% %%% Wordmall: 'Sammanfattning' (sans)
%% \textbf{Sammanfattning}

%% %%% Wordmall: 'Nyckelord' (sans) samma som 'Heading 3'?)
%% \textbf{Nyckelordsrubrik}

%% %%% Wordmall: 'Innehållsförteckning' (sans) (rubrik)


%% \subsubsection{Rubriker}

%% Alla rubriker har stor bokstav i början, men i övrigt små bokstäver (utom i
%% namn, som ju alltid har stor bokstav i början oavsett om de kommer inuti en
%% rubrik eller ej). Rubriker avslutas inte med punkt.


%% \subsubsection{Underrubriker}


%% \subsection{Färg}

%% Hela dokumentet utom fram- och baksida har svart text på vit botten.

%% På fram- och baksida har texten Stockholms universitets blåa färg
%% (\texttt{\#002F5F}).


\subsection{Referenser}
\label{referenser}

Referenserna formateras automatiskt enligt APA-standard, baserat på den
BibTeX-information du angivit i din \verb|.bib|-fil.

Placera följande kommandon på den plats i din \verb|.tex|-fil där du vill ha
dina referenser. En onumrerad rubrik på ditt valda språk kommer automatiskt
infogas före referenserna (på svenska 'Referenser').

\begin{verbatim}
\nocite{*}                               % visa även oanvända referenser
\phantomsection                          % toc point to right page
\addcontentsline{toc}{section}{\refname} % inkludera i innehållsförteckning
\bibliography{dina_referenser.bib}       % ladda 'dina_referenser.bib'
\end{verbatim}

Om du använder \verb|\nocite{*}| så kommer referenskapitlet inkludera även de
referenser som du inte refererat till i din text, men dessa visas i grått för
att göra det lättare att se vilka det är.

\verb|\addcontentsline|-kommandot lägger till referenskapitlet i din uppsats
innehållsförteckning. Notera att detta kommando måste komma precis före
\verb|\bibliography|-kommandot för att sidreferensen i innehållsförteckningen
ska peka på rätt sida.

%% [eof]

%% -*- mode: latex; ispell-dictionary: "svenska" -*-

\section{Vilken \LaTeX{} ska jag använda?}
\label{latex}

Detta är en uppsatsmall för \LaTeX{} för Stockholms universitet. Den kan
användas med \XeLaTeX{} på din egen dator eller med webbverktyget Overleaf
(\href{https://overleaf.com/edu/su}{\url{overleaf.com/edu/su}}) -- som student
på Stockholms universitet har du automatiskt tillgång Overleaf.

Denna uppsatsmall kräver \XeLaTeX{} eller \LuaLaTeX{}\footnote{Uppsatsmallen är
  inte testad med \LuaLaTeX{} men det borde fungera.} för att fungera. Den
fungerar inte med PDF\LaTeX{}. Detta beror på att bara \XeLaTeX{} och \LuaLaTeX
stödjer Truetype-typsnitt, vilket behövs för att ladda de typsnitt som
Stockholms universitets stilmanual rekommenderar.


\subsection{Overleaf}

Om du använder Overleaf måste du ändra inställningarna så att ditt dokument
kompileras med \XeLaTeX{} (om du använder det förvalda alternativet PDF\LaTeX{}
så kommer ditt dokument inte kunna kompilera).


\subsection{På egen dator}

\subsubsection{Installation av program}

Jag rekommenderar att du använder bekvämlighetsverktyget
\texttt{latexmk}\footnote{Dokumentationen till kommandot \texttt{latexmk} finns
  här: \url{https://mg.readthedocs.io/latexmk.html}. } för att kompilera din
\LaTeX{}-kod. Om du gör det kan du använda den \texttt{latexmkrc}-fil som
följer med uppsatsmallen för att om det som behövs för kompilering.

Innan du börjar behöver du se till att alla relevanta program är installerade
på datorn. Om du kör Linux (Ubuntu, Debian eller likanande) så kan du kanvända
följande kommando för att installera:

\begin{verbatim}
sudo apt install texlive-xetex latexmk
\end{verbatim}


\subsubsection{Kompilering}

Det enklaste sättet att kompilera din uppsats är:

\begin{verbatim}
latexmk su-thesis.tex
\end{verbatim}

De inställningar som behövs till \texttt{latexmk} följer med uppsatsmallen och
finns i en fil som heter \texttt{latexmkrc}. Om du installerar denna
uppsatsmall i en underkatalog så vill du förmodligen också kopiera
\texttt{latexmkrc} till katalogen där du har din uppsats.

Om du inte använder \texttt{latexmk} utan insisterar på att kompilera
själv så motsvarar detta följande kommandon (Ja! -- \texttt{xelatex} behöver
vanligtvis köras tre gånger för att vara säker på att både innehållsförteckning
och referenser stämmer överens med texten).

\begin{verbatim}
xelatex su-thesis.tex
bibtex su-thesis.bib
xelatex su-thesis.tex
xelatex su-thesis.tex
\end{verbatim}

\noindent Då och då råkar man skriva något galet i \LaTeX{}-koden, och
kompileringen stannar mitt i. Då får man upp en prompt, med ett frågetecken i
början på en ny rad. -- När detta händer skriv 'x' (och tryck enter) för att
avsluta. Rätta felet i din \texttt{.tex}-fil och kompilera om.

%% [eof]

%% -*- mode: latex; ispell-dictionary: "svenska" -*-

\section{Hur använda \LaTeX{}?}
\label{hur}

Uppsatsmallen \texttt{su-thesis} är tänkt att vara en hjälp i uppsatsskrivandet
så långt som möjligt, men det finns också andra bra vanor som inte kan byggas
in i en modul. Jag försöker här summera några av de viktigaste lärdomarna jag
gjort själv.

\LaTeX{} är populärt och har funnits lääänge (sedan 1984) vilket innebär att
det finns en uppsjö tillägg och extrapaket för allt mellan himmel och jord. Så
om det är någonting du saknar, eller funderar över -- googla!

Websajten \href{https://ctan.org/}{\url{ctan.org}} (\emph{Comprehensive Tex
  Archive Network}) har massor med dokumentation, så lägg till "ctan" i din
googlesökning om du letar efter något du gissar skulle kunna stå i
dokumentationen.


\subsection{En grundplåt}
\label{grundplåt}

Ett minimalt startdokument för din uppsats kan se ut såhär:

\begin{verbatim}
\documentclass[11pt]{article}
\usepackage[swedish]{babel}
\usepackage{su-thesis}

\suset{author}{Shevek}
\suset{supervisor}{Sabul}
\suset[swedish]{title}{Principer för samtidighet}
\suset[swedish]{subtitle}{Om tiden och samtidighetens natur}
\suset[english]{title}{Principles of Simultaneity}
\suset[english]{subtitle}{On the Nature of Time and Simultaneity}
\suset[swedish]{abstract}{...}
\suset[english]{abstract}{...}

\begin{document}
\include{kapitel-1-sekvens}      % laddar 'kapitel-1-sekvens.tex'
\include{kapitel-2-samtidighet}
\include{kapitel-3-referenser}
\end{document}
\end{verbatim}

Ovanstående dokument använder \texcommand{include} för att inkludera de olika
kapitlen. Varje deldokument som laddas på detta sätt börjar automatiskt på en
ny sida så det passar bra för just kapitel. \LaTeX{} kompilerar också de olika
deldokumenten varförsig vilket gör att det går fortare att kompilera om
helheten om man bara ändrat i en eller ett par av filerna.

Ett annat sätt man kan snabba upp kompilerandet på är genom att bara kompilera
det kapitel man för tillfället jobbar med. Om du tex arbetar med det första
kapitlet i ovanstående dokument, så kan du lägga till följande kommando
(ovanför \texcommand[document]{begin}) för att bara kompilera det första
kapitlet och kapitlet med referenser.

\begin{verbatim}
\includeonly{
  kapitel-1-sekvens,
  kapitel-3-referenser,
}
\end{verbatim}

När du sedan är klar och vill kompilera alltihop, så tar du bara bort
\texcommand{includeonly} och kompilerar om.

För mer information att välja språk se \autoref{språk}, för mer information om
vilka \texcommand{suset} variabler mallen har se \autoref{variabler}.


\subsection{Att referera inom texten}
\label{länkar}

Ibland vill man referera till en annan plats i sin egen text. Detta gör man
genom att placera ut en etikett \texcommand{label} på stället man vill referera
till. Därefter kan man skapa en länkar med kommandot \texcommand{autoref} med
samma etikett, länktexten anpassar sig automatiskt beroende på om det är en
länk till figur/tabell eller en kapitelrubrik (i det här dokumentet blir tex
\texcommand[tab-ipa]{autoref} → "\autoref{tab-ipa}" och
\texcommand[citera]{autoref} → "\autoref{citera}").

\textbf{Rekommendation:} Placera alltid ut \texcommand{label} direkt efter alla
rubriker, underrubriker och \texcommand{caption} i tabeller och figurer. Tänk
på att hålla etiketten kort men ändå unik inom ditt dokument. Det är vanligt
att låta tabeller och figurers etikett börja på "\texttt{tab-}" respektive
"\texttt{fig-}" eller liknande.


\subsection{Bilder och figurer}
\label{bilder}

Om du vill lägga till en bild i din uppsats, skapa en katalog som heter
"\texttt{images/}" och lägg bilden där. Därefter använder du \LaTeX{}-kommandot
\texcommand[FILNAMN]{includegraphics} i din uppsats för att inkludera bilden.
(Notera att filnamnet inte ska innehålla namnet på bildkatalogen
"\texttt{images/}".) En bild utgör oftast en figur i uppsatsen. En enkel figur
med en bild kan se ut såhär:

\begin{multicols}{2}
\null \vfill
\noindent\begin{verbatim}
\begin{figure}[ht]
  \centering
  \includegraphics[
    width=.666\linewidth
  ]{su-logo-sv.png}

  \caption{Bildtext till figuren.}
  \label{fig-test}
\end{figure}
\end{verbatim}

\vfill \null
\columnbreak

{
  \centering
  \hypertarget{fig-test}{%
    \includegraphics[width=.666\linewidth]{su-logo-sv.png}%
  } \\
  \vspace{.5em}
  Figur 1: Bildtext till figuren. \\ % '\\' needed for \centering to work
}
\end{multicols}

En figur börjar med \texcommand[figure]{begin}, en tabell med
\texcommand[table]{begin} och slutar med \texcommand{end}. Du behöver också
ange en bildtext med \texcommand{caption} och en etikett med
\texcommand{label}. Etiketten är ingenting som läsaren ser, utan används när du
vill referera till den i texten med kommandot \texcommand{autoref} (se
\autoref{länkar}). Notera att etiketten alltid måste komma efter bildtexten.

Figurer och tabeller är "flytande" vilket innebär att \LaTeX{} kan flytta runt
dem och placera dem där det ser bäst ut. Om en tabell eller figur inte får
plats på sidan flyttas den oftast till nästa sida. Kom därför ihåg att hänvisa
till din figur/tabell i texten. Figuren ovan har etiketten "\texttt{fig-test}"
så om vi vill referera till den använder vi kommandot
\texcommand[fig-test]{autoref} vilket resulterar i en länk med texten
"\hyperlink{fig-test}{Figur 1}".

Ibland kan en tabell eller figur växa i storlek och det kan vara praktiskt att
bryta ut den och ha den i en egen fil för att göra det lättare att navigera i
källkoden. I dessa fall brukar jag skapa en katalog "\texttt{tables/}"
och/eller "\texttt{figures/}" och lägga mina tabeller/figurer där och sedan
använda \LaTeX{}-kommandot \texcommand{input} för att läsa in tabellen där den
ska vara. (Jag brukar behålla bildtext och \texcommand{label} i
mammadokumentet, och bara använda \texcommand{input} för själva
tabellen/figuren i fråga.) Resultatet brukar se ut ungefär såhär:

\begin{verbatim}
\begin{table}
  \caption{Svenska konsonanter.}
  \label{tab-konsonanter}
  \input{tables/konsonanter}       % laddar 'tables/konsonanter.tex'
\end{table}
\end{verbatim}

\textbf{Notera:} Här använder vi kommandot \texcommand{input} till skillnad
från \texcommand{include} (som vi använde i \autoref{grundplåt}).
\texcommand{input} lägger inte till någon sidbrytning före/efter och det
kompileras om varje gång (även om ingen förändring skett) så det är bättre
lämpat för mindre deldokument (som tex tabeller/figurer) medan
\texcommand{include} passar bättre för hela kapitel.


\subsection{Att citera en referens}
\label{citera}

För att skriva en referens i din text använder antingen kommandot
\texttt{\textbackslash{}citep\{\}} (för en parentetisk referens) eller
kommandot \texttt{\textbackslash{}citet\{\}} (för en referens utan parentes
runt författarnamnet).

\begin{table}[ht]
  \centering
  \renewcommand{\arraystretch}{1.25}%
  \begin{tabular}[t]{lcc}
    \toprule
    {\sffamily\textbf{Kommando}} &
    {\sffamily\textbf{Beskrivning}} &
    {\sffamily\textbf{Exempel}} \\
    \midrule
    \texttt{\textbackslash{}citep\{källa\}} &
    med parentes &
    \citep{bergman+wallin-2001} \\
    \texttt{\textbackslash{}citet\{källa\}} &
    utan parentes &
    \citet{bergman+wallin-2001} \\
    \midrule
    \texttt{\textbackslash{}citep\{källa1,källa2\}} &
    flera källor &
    \citep{bergman+wallin-2001,bergman-1977} \\
    \texttt{\textbackslash{}citet\{källa1,källa2\}} &
    flera källor &
    \citet{bergman+wallin-2001,bergman-1977} \\
    \midrule
    \texttt{\textbackslash{}citep[1--2]\{källa1\}} &
    sidnummer &
    \citep[1--2]{bergman-1977} \\
    \texttt{\textbackslash{}citep[jmf][]\{källa1\}} &
    text före &
    \citep[jmf][]{bergman-1977} \\
    \texttt{\textbackslash{}citep[jmf][1--2]\{källa1\}} &
    text + sidref. &
    \citep[jmf][1--2]{bergman-1977} \\
    \midrule
    \texttt{\textbackslash{}citealp\{källa\}} &
    som \texttt{\textbackslash{}citep\{källa\}} &
    \citealp{bergman+wallin-2001} \\
    \texttt{\textbackslash{}citealt\{källa\}} &
    som \texttt{\textbackslash{}citet\{källa\}} &
    \citealt{bergman+wallin-2001} \\
    \bottomrule
  \end{tabular}
  \caption{Några \LaTeX{}-kommandon för att skriva referenser (från
    \texttt{natbib}-paketet)}.
  \label{tab-cite-kommandon}
\end{table}

\texttt{\textbackslash{}citealp} och \texttt{\textbackslash{}citealt}
kommandona fungerar som \texttt{\textbackslash{}citep} och
\texttt{\textbackslash{}citet} men utelämnar de omgivande parenteserna. Ibland
kan detta vara nödvändigt, som tex om man vill ange två källor med sidreferens
inom parentes. I detta fall kan man skriva
"\texttt{(\textbackslash{}citealp[55]\{källa1\} och
  \textbackslash{}citealp[25]\{källa2\})}".

Det finns ytterligare kommandon som kan användas för att skapa referenser, hur
de används finns beskrivet i
\underline{\href{http://ftp.acc.umu.se/mirror/CTAN/macros/latex/contrib/natbib/natbib.pdf}{\texttt{natbib}-paketets referensmanual}}.

%% [eof]

%% -*- mode: latex; ispell-dictionary: "svenska" -*-

\section{Att använda mallen}
\label{mallen}


\subsection{Ladda mallen}
\label{ladda}

För att ladda mallen använd \LaTeX{}-kommandot:

\begin{verbatim}
\usepackage{su-thesis}
\end{verbatim}


\subsection{Välja språk}
\label{språk}

\noindent Språket på uppsatsen ställer man in med hjälp av paketet
\texttt{babel}. När du byter språk så kommer byts fram- och baksida ut mot
information på det relevanta språket (men resten av uppsatsen får du förstås
översätta själv). :)

\begin{verbatim}
\usepackage[swedish]{babel} % Swedish
%% \usepackage[UKenglish]{babel} % British English
%% \usepackage[USenglish]{babel} % American English
\end{verbatim}

\noindent\textbf{Notera:} När du bytt språk behöver du radera alla tempfiler
och kompilera om. Om du använder \texttt{latexmk} är detta enkelt.

\begin{verbatim}
latexmk -c su-thesis.tex  % radera alla tempfiler
latexmk su-thesis.tex     % kompilera om
\end{verbatim}


\subsection{Färger}
\label{färg}

Uppsatsmallen ger dig tillgång till följande färger kommer från Stockholms
universitets grafiska profil. Använd kommandot \texcommand[NAMN]{color} med
färgens namn för att byta färg på texten (tex används
\texcommand[sublue]{color} för att sätta färg på uppsatsens fram- och
baksida).

\medskip

\begin{center}
  \begin{tabular}{ccc}
    \toprule
    \thead{namn} & \thead{färg} & \thead{hexkod} \\
    \midrule
    sublue  &  \color{sublue}\rule{3em}{1.5em} & \texttt{\#002F5F} \\
    suolive & \color{suolive}\rule{3em}{1.5em} & \texttt{\#A3A86B} \\
    susky   &   \color{susky}\rule{3em}{1.5em} & \texttt{\#ACDEE6} \\
    suwater & \color{suwater}\rule{3em}{1.5em} & \texttt{\#9BB2CE} \\
    sufire  &  \color{sufire}\rule{3em}{1.5em} & \texttt{\#D95E00} \\
    \bottomrule
  \end{tabular}
\end{center}

\medskip

Färgerna är definierade med \LaTeX{}-paketet "\texttt{xcolor}", så för mer
information om hur du kan använda dem, se dokumentationen du hittar här:
\url{https://ctan.org/pkg/xcolor}.


\subsection{Mallens variabler}
\label{variabler}

Mallen har ett antal variabler som man sätter med \verb|\suset[SPRÅK]{NAMN}|
(för tillfället kan bara \texttt{swedish} och \texttt{english} användas som
språk).

\textbf{Notera:} Du behöver ladda mallen innan du kan sätta variablerna.

\begin{verbatim}
%% Always used
\suset{author}{Förnamn Efternamn}
\suset{supervisor}{Förnamn1 Efternamn1[, Förnamn2 Efternamn2]}
\suset[swedish]{title}{Titel på svenska}
\suset[english]{title}{Title in English}
\suset[swedish]{abstract}{Sammanfattning på svenska}
\suset[english]{abstract}{Abstract in English}

%% Swedish metadata (only used if language is Swedish)
\suset[swedish]{subtitle}{Svensk undertitel}
\suset[swedish]{keywords}{nyckelord1, nyckelord2, nyckelord3, \ldots}
\suset[swedish]{department}{Institutionen för lingvistik}
\suset[swedish]{thesistype}{Examensarbete 15 hp}
\suset[swedish]{course}{Lingvistik -- kandidatkurs, LIN600}
\suset[swedish]{program}{Kandidatprogrammet i lingvistik 180 hp}
\suset[swedish]{semester}{Vårterminen 2020}

%% English metadata (only used if language is English)
\suset[english]{subtitle}{English Subtitle}
\suset[english]{keywords}{keyword1, keyword2, keyword3, \ldots}
\suset[english]{department}{Department of Linguistics}
\suset[english]{thesistype}{Bachelor's Thesis 15 ECTS credits}
\suset[english]{course}{Linguistics -- Bachelor's Course, LIN600}
\suset[english]{program}{Bachelor's Programme in Linguistics 180 ECTS credits}
\suset[english]{semester}{Spring semester 2020}
\end{verbatim}


\subsection{Fonetisk transkription (IPA)}
\label{ipa}

Den version av Times New Roman som kommer med uppsatsmallen\footnote{Times New
  Roman har haft stöd för IPA-symboler sedan version 5.22 \citep{ipa} som kom
  med Windows 7 \citep{win-7-fonts}. I uppsatsmallen är Times New Roman version
  7.00 och Verdana 5.22 inkluderade (från Windows 10).} har stöd för
IPA-symboler (International Phonetic Alphabet). Detta fungerar med både fet och
kursiv stil fungerar, men dessvärre finns inte IPA-symbolerna med i Verdana, så
det går bara att få fonetiska symboler med serifftypsnitt. Tabell
\ref{tab-ipa} är ett exempel på hur det kan se ut.

Om du saknar en inmatningsmetod för att skriva IPA-symboler så kan du använda
följande hemsida och därefter klippa/klistra in resultatet i din uppsats:
\url{https://ipa.typeit.org/full/}. (Det går förstås lika bra att använda Word,
eller andra program och klistra in resultatet därifrån i ditt dokument.)

\begin{table}
  \caption{IPA-exempel, svenska konsonanter (från \citealp[140]{engstrand-1999},
    egen översättning).}
  \label{tab-ipa}
  \vspace{.5em}
  %%
%% This table uses 'tabularx', 'array' and 'makecell' packages.
%%
%% * 'Tables in LATEX2ε: Packages and Methods' (docs on tables etc.)
%%   https://www.tug.org/pracjourn/2007-1/mori/mori.pdf
%%
%% * 'The tabularx package∗'
%%   http://mirrors.ibiblio.org/CTAN/macros/latex/required/tools/tabularx.pdf
%%
%%
%% Random Observations
%% ===================
%%
%%       \setlength{\extrarowheight}{40pt}%   % array:
%%       \setlength{\tabcolsep}{<length>}     % tabular: ½ of width between cols
%%       \setlength{\arraycolsep}{<length>}   % array:   ½ of width between cols
%%       %%\extracolsep{\fill}
%%
%%   Vertical padding is possible in a global way using @Herbert's answer. That
%%   is, to redefine the array stretch factor <factor> using
%%
%%       \renewcommand\arraystretch{.01}%      % (array)
%%
%%   \baselineskip -- Specifies minimum space between baselines successive
%%   lines in a paragraph. Is reset, for example, by font changes. The value in
%%   effect at end of a paragraph, is used for whole paragraph.
%%
%%       \setlength{\baselineskip}{0pt}%      % minimum line height (reset by font commands)
%%
%%   \baselinestretch is used to multiply \baselineskip. (Default: 1.0). Use
%%   this to change line height in a document, since its not reset by other
%%   commands.
%%
%%       \renewcommand{\baselinestretch}{0}%   % multiplier of line height (default: 1)
%%       \arraybackslash%                      % '\\' inside table
%%
%%
%% Column Specifications
%% =====================
%% These column specifications are used by 'tabularx'. Some of them are
%% imported from the 'array' package, and some from the (non-x) 'tabular'.
%%
%%     @{text}     Replace padding between this & previous column with <text>
%%     >{code}     (array) insert <code> at beginning of cell
%%     l           left aligned
%%     r           right aligned
%%     c           horizontally centered
%%     p{width}    justified
%%     m{width}    (array) vertically centered
%%     b{width}    (array) bottom align
%%     <{code}     (array) insert <code> at end of cell
%%     @{text}     Replace padding between this & next column with <text>
%%
%% Relative Column Widths ('tabularx')
%% ===================================
%% If there are two columns, then then all <\hsize>s added together should be
%% 2, but if you want differing widths you could use something like this (which
%% will cause the first column to be three times wider than the second):
%%
%%    \begin{tabularx}{\linewidth}{|%
%%      >{\hsize=1.5\hsize\linewidth=\hsize}c|%
%%      >{\hsize=0.5\hsize\linewidth=\hsize}c|}
%%
\centering%
{
  \large%
  \newcommand{\na}{\makecell{}}               % empty cell content
  \newcommand{\td}[1]{\makecell{#1}}
  \renewcommand{\theadfont}{\normalsize}      % small font in header
  \renewcommand\theadalign{cc}
  \renewcommand\cellalign{cc}
  \renewcommand\theadset{                     % (makecell)
    \renewcommand\arraystretch{.6}%           % (array) line height
    %\renewcommand\theadalign{c}
  }%
  \renewcommand{\tabularxcolumn}[1]{>{}m{#1}}
  \newcolumntype{Z}{%                       % phoneme column
    >{\centering\arraybackslash}X%
  }%
  \newcolumntype{P}{%                       % Placeholder column
    @{}%                                    %   suppress left margin
    c%                                      %   horizontally centered text
    @{}%                                    %   suppress right margin
  }%
  %% Left side table headers.
  %%
  %% Leading & trailing space should be of same width here. But something is
  %% adding approx .2em space on the left side, so that's from the left hand
  %% side subtracted below. (Specifying an empty @{} will suppress the
  %% default space between columns.)
  %%
  \newcolumntype{H}{%             % Header
    @{\hspace{.3em}}%             %   left margin
    c%                            %   horizontally centered text
    @{\hspace{.5em}}%             %   right margin
  }%
  \begin{tabularx}{\linewidth}{P|H|ZZ|ZZ|ZZ|ZZ|ZZ|ZZ|ZZ|}
    \hline%---------------------------------------------------------------------
    \makecell{\rule{0pt}{1.5em}} &
                                                & % header = 3 columns wide
    \multicolumn{2}{c|}{\thead{Bilabial}}       &
    \multicolumn{2}{c|}{\thead{Labio-\\dental}} &
    \multicolumn{2}{c|}{\thead{Dental}}         &
    \multicolumn{2}{c|}{\thead{Alveolar}}       &
    \multicolumn{2}{c|}{\thead{Palatal}}        &
    \multicolumn{2}{c|}{\thead{Velar}}          &
    \multicolumn{2}{c|}{\thead{Glottal}} \\
    \hline%---------------------------------------------------------------------
    \makecell{\rule{0pt}{1.5em}} &
    \thead{Klusil} &
    \td{p} & \td{b} &  % bilabial
    \na    & \na    &  % labiodental
    \td{t} & \td{d} &  % dental
    \na    & \na    &  % alveolar
    \na    & \na    &  % palatal
    \td{k} & \td{ɡ} &  % velar
    \na    & \na    \\ % glottal
    \hline%---------------------------------------------------------------------
    \makecell{\rule{0pt}{1.5em}} &
    \thead{Nasal} &
    \na & \td{m} &  % bilabial
    \na & \na    &  % labiodental
    \na & \td{n} &  % dental
    \na & \na    &  % alveolar
    \na & \na    &  % palatal
    \na & \td{ŋ} &  % velar
    \na & \na    \\ % glottal
    \hline%---------------------------------------------------------------------
    \makecell{\rule{0pt}{1.5em}} &
    \thead{Frikativa} &
    \na    & \na    &  % bilabial
    \td{f} & \td{v} &  % labiodental
    \td{s} & \na    &  % dental
    \na    & \na    &  % alveolar
    \na    & \td{ʝ} &  % palatal
    \na    & \na    &  % velar
    \td{h} & \na    \\ % glottal
    \hline%---------------------------------------------------------------------
    \makecell{\rule{0pt}{1.5em}} &
    \thead{Approximant} &
    \na & \na    &  % bilabial
    \na & \na    &  % labiodental
    \na & \na    &  % dental
    \na & \td{ɹ} &  % alveolar
    \na & \na    &  % palatal
    \na & \na    &  % velar
    \na & \na    \\ % glottal
    \hline%---------------------------------------------------------------------
    \makecell{\rule{0pt}{1.5em}} &
    \thead{Lateral\\approximant} &
    \na & \na    &  % bilabial
    \na & \na    &  % labiodental
    \na & \td{l} &  % dental
    \na & \na    &  % alveolar
    \na & \na    &  % palatal
    \na & \na    &  % velar
    \na & \na    \\ % glottal
    \hline%---------------------------------------------------------------------
  \end{tabularx}

  \medskip

  ɧ {\normalsize \hspace{.25em} Tonlös dorso-palatal/velar frikativa}
  \hspace{.8em}
  ɕ {\normalsize \hspace{.25em} Tonlös alveolar-palatal frikativa}
}

%% [eof]

  \vspace{1em}
\end{table}


\subsection{Teckenspråkstranskription}
\label{teckenspråk}

Med uppsatsmallen kommer sansserifftypsnittet FreeSans-SWL ("SWL" är ISO-koden
för svenskt teckenspråk) som innehåller de teckentranskriptionssymboler som
används för svenskt teckenspråk. (Symbolerna används inte utanför Sverige och
det finns dessvärre ingen version av Verdana eller Times New Roman som
inkluderar dem.)

Det saknas stöd för teckentranskriptionssymbolerna i de flesta
textredigeringsprogram, så med största sannolikhet kommer du inte att kunna se
dem när du jobbar med texten. (Om du vill försöka kan du kopiera typsnittet
från "\texttt{fonts/freesans-swl.ttf}" och installera det på din dator och
därefter testa att välja det som typsnitt i ditt textredigeringsprogram.)

Det enklaste sättet att skriva symbolerna är att använda följande webbsida, och
därefter klistra in resultatet därifrån in i ditt \LaTeX{}-dokument:
\href{https://zrajm.github.io/teckentranskription/}{\url{zrajm.github.io/teckentranskription/}}

Det finns två \LaTeX{} kommandon för att använda det: \texcommand{swl} och
\verb|\swlfamily|. \texcommand{swl} är i det flesta situationer det mer
praktiska av de två. Ange den text du vill visa med FreeSans-SWL som argument
(vanligtvis är detta transkriptionen, men FreeSans-SWL är ett Unicode-typsnitt
och har support för de flesta språk).

\medskip

\hspace{1em}\texcommand[????????????????]{swl} → \swl{􌥃􌥔􌥘􌥃􌤵􌤷􌥧􌥡􌥼􌥲􌦊􌥱􌦈􌥼􌤟􌥣}

\medskip

\verb|\swlfamily| byter typsnitt till FreeSans-SWL och passar bättre till
längre texter (jag har aldrig sätt en längre text skriven på detta sätt). För
att byta tillbaka till originaltypsnittet kan du antingen använda
\verb|{\swlfamily …}|, eller använda \verb|\rmfamily| för att byta tillbaka
till Times New Roman efteråt, eller använda \verb|\sffamily| för att byta till
Verdana.

\begin{verbatim}
\swlfamily
Teckenspråk heter ???????????????? på teckenspråk.
\rmfamily
\end{verbatim}

\swlfamily
\hspace{1em}→ Teckenspråk heter 􌥃􌥔􌥘􌥃􌤵􌤷􌥧􌥡􌥼􌥲􌦊􌥱􌦈􌥼􌤟􌥣 på teckenspråk.
\rmfamily


\subsection{Ta bort inledande/avslutande sidor}
\label{genererade}

Mallen använder interna kommandon för att genera inledande och avslutande
sidor. För att ta bort en eller flera av dessa kan man definiera om de
relevanta kommandona så att det inte generar någon effekt. Detta görs med hjälp
av \LaTeX{}-kommandot \verb|\renewcommand|. Följande kommandon kan användas:

\begin{verbatim}
\usepackage{su-thesis}
...
\renewcommand{\sumakefrontpage}{}     % ta bort framsida
\renewcommand{\sumakeabstractpage}{}  % ta bort sammanfattningssida
\renewcommand{\sutableofcontents}{}   % ta bort innehållsförteckning
\renewcommand{\sumakebackpage}{}      % ta bort baksida
\end{verbatim}

\noindent\textbf{Notera:} Raden med \verb|\renewcommand| måste utföras efter
\verb|\usepackage{su-thesis}|.

%% [eof]

%% -*- mode: latex; ispell-dictionary: "svenska" -*-

\section{Strukturen på en uppsats}
\label{struktur}

Nedan försöker jag beskriva praxis vid Institutionen för lingvistik på
Stockholms universitet, så som jag uppfattat den. Detta bygger delvis på
jämförelser mellan olika kandidatuppsatser, delvis på information från
\citet{schott+others-2007, wiren-2020} och \citet{uppsatsguide-2020}.

\textbf{Längd:} En typisk kandidatuppsats är 20--50 sidor och en magister-
eller masteruppsats är 30--60 sidor.

\begin{itemize}
\item0. \nameref{rubrik.sammanfattning}
\item1. \nameref{rubrik.inledning}
\item2. \nameref{rubrik.bakgrund}
\item3. \nameref{rubrik.syfte}
\item4. \nameref{rubrik.metod}
\item5. \nameref{rubrik.resultat}
\item6. \nameref{rubrik.diskussion}
\item7. \nameref{rubrik.slutsats}
\item\nameref{rubrik.referenser}
\item (Eventuella bilagor)
\end{itemize}

\noindent Namnsättningen på eventuella underrubriker är mestadels friare men
några underrubriker återkommer dock oftare än andra, och i förekommande fall
finns detta beskrivet nedan.

Exakt vilka rubriker som används och hur de är underordnade varandra varierar
från uppsatsguide till uppsatsguide (ofta inkluderas tex rubrikerna
\emph{\nameref{rubrik.bakgrund}} och \emph{\nameref{rubrik.syfte}} under
\emph{\nameref{rubrik.inledning}}). Det som beskrivs nedan är den vanligaste
uppsättningen rubriker för uppsatser vid Institutionen för Lingvistik, på
Stockholms universitet.


\setcounter{subsection}{-1}% start subsection numbering from zero.
\subsection{Sammanfattning och nyckeord}
\label{rubrik.sammanfattning}

\textbf{Sammanfattning} (eller '\textbf{abstract}' på engelska) är en max 200
ord lång text, bestående av endast ett stycke (utan underrubriker). Syftet med
sammanfattningen är att kortfattat beskriva hela uppsatsen (och dess resultat)
på ett sådant sätt att en ska läsare kunna avgöra om det är intressant att
fortsätta och läsa hela uppsatsen.

En sammanfattningen måste finnas på både svenska och engelska i din uppsats
(oavsett om du skriver på svenska eller engelska). Använd följande kommandon
(före \texcommand[document]{begin} i ditt \LaTeX-dokument) för att lägga till
en sammanfattning till din uppsats:

\begin{verbatim}
\suset[english]{abstract}{…}
\suset[swedish]{abstract}{…}
\end{verbatim}

\textbf{Nyckelord} ('\textbf{keywords}' på engelska) är vanligtvis 4--6 termer,
avsedd att ge en hyperkondenserad inblick i uppsatsens ämne. Nyckelord sägs
vara viktiga, då de ökar sannolikheten att din tänkta publik hittar din
uppsats, men i ärlighetens namn undrar jag om det fortfarande gäller i detta
tidevarv då alla sökmotorer tycks indexera artiklars hela text. -- Hursomhelst,
av gammal hävd ska det finnas nyckelord i din uppsats! Bra nyckelord är längre
fraser som beskriver ditt uppsatsämne. (Fraser som förekommer ofta i din
uppsats kan vara lämpliga nyckelord.)

Följande kommandon (före \texcommand[document]{begin} i ditt dokument) för att
lägga till en nyckelord till din uppsats:

\begin{verbatim}
\suset[english]{keywords}{…}
\suset[swedish]{keywords}{…}
\end{verbatim}


\subsection{Inledning}
\label{rubrik.inledning}

Inledningen är oftast ungefär ¼--1 sida lång, och brukar inte innehålla några
underrubriker.

Här berättar du varför din forskningsfråga är viktig. – Inledningen kan
beskrivas som en tratt, i det att den börjar brett och sedan smalnar av till
att handla om just din undersökning, ditt ämne.

Inledningen kan ta sitt avstamp i något dagsaktuellt, eller något större (som
till exempel ett samhällsproblem), men bör inte grundas i vad du personligen
tycker är intressant. Inledningen ska heller inte börja alltför långt ifrån
ämnet.


\subsection{Bakgrund}
\label{rubrik.bakgrund}

Brukar vara 2--9 sidor lång (vanligast är cirka 6 sidor) och har ofta
underrubriker specifika för ämnet. Bakgrunden kan ses som en sort förlängning
av inledningen som går in i mer detalj om rådande forskningsläge, och ger
läsaren den bakgrundskunskap och förankring som behövs för att kunna ta till
sig din undersökning.

Syftet med bakgrundskapitlet är att peka ut ett problemområde, beskriva
tidigare forskning och det aktuella forskningsläget inom området (med
åtföljande referenser), samt identifiera en kunskapslucka som utgör motivering
till varför undersökningen ska göras.

Här presenteras utgångspunkten och det allmänna syftet. Vad är motivet och
ämnesvalet? Varför är just detta viktigt och intressant? Vad är bakgrunden till
det som skall studeras? Här kan man också gå igenom uppsatsens disposition och
huvuddrag. Vad handlar de olika kapitlen om? Viktiga termer och begrepp kan
också presenteras här. Tänk dock på att inte ha inte mer bakgrundsinformation
än nödvändigt. En onödigt lång bakgrund gör inte din text bättre.


\subsection{Syfte och frågeställningar}
\label{rubrik.syfte}

En mycket kort beskrivning av uppsatsens centrala frågeställningar. Vanligtvis
4–7 \emph{rader} lång (ett eller två korta stycken). Denna kan växa och bli
något längre särskilt om författaren väljer att ställa upp sina frågor i form
av en punktlista, men även i dessa fall har jag inte sett någon uppsats i
vilken denna del var längre än en halv sida.

Frågeställningarna är oftast formulerade som frågor, men kan också formuleras
som "syftet med studien är att beskriva..." eller liknande formuleringar. Syfte
och frågeställningar är den mest centrala delen i uppsatsen, och styr mycket
hur de övriga delarna kommer se ut.


\subsection{Metod och material}
\label{rubrik.metod}

1--8 sidor, men vanligast är 1--2 sidor. Beskriver hur studien gjorts, och
motiverar varför du gjort som du gjort.

Om kapitlet är längre är det oftast indelat i underrubriker, med en underrubrik
vardera för de olika korpusar eller datainsamlingsmetoder som används. Här
beskrivs också ofta vilka avgränsningar som gjorts, etiska aspekter vid
intervju eller korpusanvändande, tekniska begränsningar, hur intervju eller
enkät sett ut, eller tänkbar skevhet eller partiskhet i data eller metod.


\subsection{Resultat}
\label{rubrik.resultat}

5--30 sidor, vanligast cirka 15 sidor. Redovisar vad du kommit fram till och
redovisar det resultat som är relevant för syftet med studien. Kapitlet är
alltid indelat i underkapitel, och oftast också med under-underrubriker.


\subsection{Diskussion}
\label{rubrik.diskussion}

1--7 sidor, men vanligast är 2--4 sidor. Diskuterar hur man kan se på
resultatet utifrån olika synvinklar och kopplar resultatet till den tidigare
forskningen. Har alltid underrubriker. Vanliga underrubriker är
"resultatdiskussion", "metoddiskussion", "etikdiskussion" och "framtida
forskning", men det än inte heller ovanligt med andra underrubriker specifika
för ämnet. Under-underrubriker är också vanliga.


\subsection{Slutsatser}
\label{rubrik.slutsats}

Överstiger oftast inte en sida, utom i längre uppsatser. Underrubriker
förekommer inte.


\subsection{Referenser}
\label{rubrik.referenser}

Ett onumrerat kapitel. Brukar vara 1--2 sidor (oftast ganska precis en sida) i
längd. Och innehåller bara en lista av de referenser refererats till i
uppsatsens övriga text. Om du i ditt \LaTeX{}-dokument valt att visa alla
referenser (med kommandot \verb|\nocite{*}|) så ser uppsatsmallen till så att
de referenser du ännu inte refererat till i texten markeras med grå text
(istället för svart).

%% [eof]

\nocite{*}                               % show all references (even non-cited)
\phantomsection                          % make toc point here (hyperref)
\addcontentsline{toc}{section}{\refname} % include section in table of content
\renewcommand{\bibliographyprenote}{

  I vanliga fall så innehåller referenskapitlet endast de referenser som du
  faktiskt används i din huvudtext, men om du istället vill inkludera alla
  referenser som finns i din \texttt{.bib}-fil (vare sig de används eller ej)
  så kan du använda kommandot \texcommand[*]{nocite}. För att hjälpa dig att
  hålla reda på om du har ociterade referenser så markerar uppsatsmallen dessa
  med grå text (se nedan -- för att visa hur det ser ut är en av referenserna
  nedan oanvänd).

  \medskip
}
\bibliography{su-thesis-6-referenser.bib}

%% [eof]

%% \begin{appendices}
%% \include{su-thesis-9-bilagor}
%% \end{appendices}
\end{document}

%% [eof]

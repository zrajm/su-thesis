%% -*- mode:latex; ispell-dictionary: "svenska" -*-
\documentclass[11pt]{article}

%% Nowidow's documentation:
%% http://ftp.acc.umu.se/mirror/CTAN/macros/latex/contrib/nowidow/nowidow.pdf
\usepackage[all]{nowidow} % Tries to remove widows

%% Microtype's documentation:
%% http://ftp.acc.umu.se/mirror/CTAN/macros/latex/contrib/microtype/microtype.pdf
\usepackage{microtype} % Improves typography, load after fontpackage is selected

\usepackage[swedish]{babel} % Swedish
%% \usepackage[UKenglish]{babel} % British English
%% \usepackage[USenglish]{babel} % American English

\usepackage{su-thesis}

%% Always used
\suset{author}{Zrajm}
\suset{supervisor}{Ingen}
\suset[swedish]{title}{Mall för uppsats vid Stockholms~universitet}
\suset[swedish]{subtitle}{För {\rmfamily\LaTeX} och Overleaf}
\suset[english]{title}{Thesis Template for Stockholm~University}
\suset[english]{subtitle}{For {\rmfamily\LaTeX} and Overleaf}
\suset[swedish]{abstract}{
  Detta är en introduktion till hur man använder Stockholms universitets
  uppsatsmall för \LaTeX. Här går vi igenom fyra saker: (a) Hur Stockholms
  universitets uppsatsmall ser ut. (b) Vilka möjligheter du, som student vid
  Stockholms universitet, har att använda \LaTeX. (c) Hur denna \LaTeX-version
  av uppsatsmallen används. Och slutligen (d) går vi i grova drag igenom
  strukturen på hur en uppsats på kandidat- eller masternivå brukar se ut på
  Institutionen för lingvistik. Denna text tjänar också som ett exempel på en
  text skriven i \LaTeX och är tänkt att kunna användas som utgångspunkt vid
  uppsatsskrivande.
}
\suset[english]{abstract}{
  This is an introduction to Stockholm university's thesis template for \LaTeX.
  In this guide we will cover four things: (a) What thesis template of
  Stockholm university looks like. (b) What choices you, as a student at
  Stockholm university, have for running \LaTeX. (c) How to use this \LaTeX
  version of the template. And, finally, we also cover roughly (d) what the
  expected structure of a bachelor's or master's thesis look like at the
  Department of Linguistics. This text also serve as an example of a text
  written in \LaTeX and is indended as a possible starting point when writing a
  thesis.
}

%% Swedish metadata (only used if language is Swedish)
\suset[swedish]{keywords}{
  Stockholms universitet, uppsatsmall, nybörjarguide till \LaTeX, kandidatuppsat, masteruppsats
}
\suset[swedish]{department}{Institutionen för lingvistik} % e.g. ''
\suset[swedish]{thesistype}{Examensarbete X hp/Degree X HE credits} % e.g. 'Examensarbete 15 hp'
\suset[swedish]{course}{Examensämne} % e.g. 'Lingvistik -- kandidatkurs, LIN600'
\suset[swedish]{program}{Kandidatprogrammet i lingvistik 180 hp}
\suset[swedish]{semester}{Vårterminen 2020}

%% English metadata (only used if language is English)
\suset[english]{keywords}{
  Stockholm university, thesis template, beginner's guide to \LaTeX, bachelor's
  thesis, master's thesis
}
\suset[english]{department}{Department of Linguistics}
\suset[english]{thesistype}{Bachelor's Thesis 15 ECTS credits}
\suset[english]{course}{Thesis Subject} % e.g. 'Linguistics -- Bachelor's Course, LIN600'
\suset[english]{program}{Bachelor's Programme in Linguistics 180 ECTS credits}
\suset[english]{semester}{Spring semester 2020}

%% Use 'apacite' package for references
\usepackage[natbibapa]{apacite}
\setcitestyle{notesep={:}}    % put ':' between year and page number
\apptocmd{\thebibliography}   % remove extra space after period in references
         {\def\newblock{}}    % [https://tex.stackexchange.com/questions/480787]
         {}{}

%% Hyphenation patterns
\hyphenation{Fo-ne-tik-lab-bet}

%% Remove margin around lists
\usepackage{enumitem}
\setlist{itemsep=0pt}

%% For XeLaTeX logo
\usepackage{metalogo}

%% Suppress abstract page.
%\renewcommand{\sumakeabstractpage}{}

\usepackage{tabularx}
\usepackage{longtable}

%%%%%%%%%%%%%%%%%%%%%%%%%%%%%%%%%%%%%%%%%%%%%%%%%%%%%%%%%%%%%%%%%%%%%%%%%%%%%%%%
\begin{document}

%% -*- mode: latex; ispell-dictionary: "svenska" -*-

\section{Stockholms universitets uppsatsmall}
\label{utseende}

Denna uppsatsmall bygger delvis på Stockholms universitets studentuppsatsmall
för Word, delvis på andra stilrekommendationer från Stockholms universitet som
jag kunnat hitta. Jag har också undersökt och jämfört ett tiotal
kandidatuppsatser skrivna vid institutionen för lingvistik och därifrån hämtat
inspiration till sådant som jag inte hittat beskrivet.

\medskip

\begin{itemize}
\raggedright
\item Stockholms universitetsbiblioteks studentuppsatsmall (finns bara till
  Word):\\
  \url{https://su.se/biblioteket/guider/guider/studentuppsatsmall}

\item Stockholms universitetsbiblioteks "Mallar och visuell identitet för
  avhandling" \\
  (Word-mallar för olika typer av avhandlingar): \\
  \url{http://su.se/biblioteket/avhandlingsmallar}

\item Stockholms universitetsbiblioteks "Dokumentmall i Word för
  doktorsavhandlingar" (PDF): \\
  \url{https://su.se/polopoly\_fs/1.264229.1576661286!/menu/standard/
    file/Instruktioner%20SU\_Wordmall\_20180509.pdf}

\item Institutionen för svenska och flerspråkighets "Uppsatsmall och
  Word-tips": \\
  \url{https://su.se/svefler/publikationer/uppsatsarkiv/
    uppsatsmall-och-word-tips}

\item Stockholms universitets grafiska manual (allmän information om hur
  grafisk profil, ingenting specifikt om utseende på uppsatser/avhandlingar):
  \\
  \url{https://su.se/medarbetare/kommunikation/grafisk-manual}
\end{itemize}

%% \subsection{Typsnitt}

%% Brödtexten är satt i Times New Roman, medan titel, rubrik, underrubriker etc.
%% är satta i Verdana. På fram- och baksida är all text och universitetets logotyp
%% satta i Stockholms universitetets blåa färg, på övriga sidor är all text svart.

%% %% Brödtext (12 uppsatser):
%% %%  Blankrad: 11 / Indrag: 1 (Bark 2018)
%% %%  Vänsterjusterad: 7, Spärrad: 5 (Bäckström 2015, Lyxell 2016, Petersdotter
%% %%                                 2018, Bark 2018, Klintberg 2018)

%% \subsubsection{Times New Roman}

%% Brödtexten används för den huvudsakliga texten i uppsatsen. Den används också
%% för texten i styckena för sammanfattning/abstract, och nyckelord/keywords. Den
%% används däremot \emph{inte} för innehållsförteckning eller i textrutorna på
%% fram- och baksida.

%% %\textbf{Styckeindrag} -- Inget.

\subsection{Textstilar}

För att \emph{kursivera text} använd \texcommand{emph}. För \textbf{fetstil}
använd \texcommand{textbf}. Om du vill skriva med \texttt{skriv\-maskins\-stil}
(vilket ofta används för variabelnamn eller kod i text om programmering) använd
\texcommand{texttt}. Du kan också \underline{stryka under text} med
\texcommand{underline}. Det finns många andra \LaTeX{}-stilkommandon du också
kan använda men dessa är de vanligaste.

\LaTeX{} är oftast rättså bra på att luska ut hur ord ska avstavas men ibland
kan det gå snett och ett ord kan sticka långt ut i marginalen. För att fixa
detta när det händer det kan du antingen skriva \verb|\-| inuti ordet där du
vill tillåta avstavning (dvs \verb|Fo\-ne\-tik\-lab\-bet|). Alternativt så kan
du lägga till ett skapa en lista i början av ditt dokument (den måste komma
före \verb|\begin{document}|) med alla avstavningsmönster du behöver.

\begin{verbatim}
\hyphenation{Fo-ne-tik-lab-bet}
\hyphenation{kort-films-festi-vals-med-ar-betar-fest}
\end{verbatim}

En fördel med att lägga till avstavningsmönster i början av dokumentet är att
om ett ord förekommer flera gånger så behöver du bara beskriva avstavningen på
ett ställe. En annan fördel är att du kan söka på ordet i din text och inte
måste komma ihåg att just det ordet var avstavat.

För att skriva webbadresser i din text använd kommandot \texcommand{url}. Det
ser ut såhär i text \url{http://www.exempel.com/}, skapar en länk i texten och
hanterar eventuella radbrytningar på ett bra sätt (utan att stoppa in
bindestreck som förändrar vart adressen pekar).


\subsection{Rubriker och brödtext}

Sidmarginalerna är 2,5~cm på alla fyra sidor, utom upptill där den är 2,725~cm.
Rubriker är vänsterjusterade och satta med Verdana, medan brödtext har rak
högermarginal och är satt med Times New Roman. Stycken är separerade med ett
vertikalt mellanrum utan styckeindrag.

\medskip

\begin{tabular}{p{.33\textwidth} p{.67\textwidth}}
  \toprule
  \thead{Typ} &
  \thead{Beskrivning} \\
  \midrule
%%% Wordmall: 'Heading 1' (sans)
  \thead{
    \Large\sffamily\mdseries
    1~Rubrik
  } &
  \makecell[{{p\linewidth}}]{
    \texcommand{section} \\
    Numrerad kapitelrubrik. Verdana 26~punkter. Vänsterjusterad.
    % I Libreoffice: (Line spacing "at least" 15~punkter.)
    Avstånd ovanför 36~punkter, avstånd nedanför 24~punkter.
  } \\
  \midrule
%%% Wordmall: 'Heading 2' (sans) -- \subsection{}
  \thead{
    \large\sffamily\mdseries
    1.1~Underrubrik
  } &
  \makecell[{{p\linewidth}}]{
    \texcommand{subsection} \\
    Numrerad underrubrik. Verdana 16~punkter fetstil. Vänsterjusterad.
    % I Libreoffice: (Line spacing "at least" 15~punkter.)
    Avstånd ovanför 24~punkter, avstånd nedanför 6~punkter.
  } \\
  \midrule
%%% Wordmall: 'Heading 3' (sans) -- \subsubsection{}
  \thead{
    \normalsize\sffamily\bfseries
    1.1.1~Under-underrubrik
  } &
  \makecell[{{p\linewidth}}]{
    \texcommand{subsubsection} \\
    Numrerad under-underrubrik. Verdana 11~punkter fetstil. Vänsterjusterad.
    % I Libreoffice: (Line spacing "at least" 15~punkter.)
    Avstånd ovanför 12~punkter, avstånd nedanför 6~punkter.
  } \\
  \midrule
%%% Wordmall: 'Default Style'
  %% Set with:
  %%   \documentclass[11pt]{article} % set base font size
  %%   \setlength{\parindent}{0cm}   % remove 1st line of paragraph indentation
  %%   \setlength{\parskip}{6pt}     % add vertical space between paragraphs
  \thead{
    \normalsize\rmfamily\mdseries
    Brödtext
  } &
  \makecell[{{p\linewidth}}]{
    Times New Roman 11~punkter. Vänsterjusterad.
    % I Libreoffice: (Line spacing: Single.)
    Avstånd ovanför 0~punkter, avstånd nedanför 6~punkter.
  } \\
  \bottomrule
%% %%% Wordmall: 'Header'
%%   Sidhuvud &
%%   \hl{FIXME}
%%   \\
%%   \midrule
%% %%% Wordmall: 'Footer'
%%   Sidfot &
%%   \hl{FIXME}
%%   \\
%%   \midrule
\end{tabular}


\subsection{Fram- och baksida}

Texten på fram- och baksidor är helt och hållet satt med Verdana (dvs
sansseriff) och har Stockholms universitets blå färg (på både text och
logotyp).

\medskip

\begin{tabular}{p{.33\textwidth} p{.67\textwidth}}
  \toprule
  \thead{Typ} &
  \thead{Beskrivning} \\
  \midrule
%%% Wordmall: 'Uppsatstitel' (sans, blå)
  \thead{
    \color{sublue}\LARGE\sffamily\mdseries
    Titel
  } &
  \makecell[{{p\linewidth}}]{
    \texcommand{suset[LANG]\{title\}} \\
    Uppsatstiteln på framsidan. Verdana 28~punkter rak. Vänsterjusterad.
    % I Libreoffice: (Line spacing "at least" 15~punkter.)
  } \\
  \midrule
%%% Wordmall: 'Undertitel' (sans, blå)
  \thead{
    \color{sublue}\subtitlesize\sffamily\mdseries
    Undertitel
  } &
  \makecell[{{p\linewidth}}]{
    \texcommand{suset[LANG]\{title\}} \\
    Undertitel och författarnamn på framsidan. Verdana 14~punkter rak.
    Vänsterjusterad.
    % I Lireoffice: (Line spacing "at least" 24~punkter.) Avstånd ovanför
    % 6~punkter, avstånd nedanför 28~punkter.
  } \\
  \midrule
%%% Wordmall: 'Textruta' (sans, blå)
  \thead{
    \color{sublue}\small\sffamily\mdseries
    Textruta
  } &
  \makecell[{{p\linewidth}}]{
        \texcommand{suget[LANG]\{department\}} \\
        \texcommand{suget[LANG]\{thesistype\}} \\
        \texcommand{suget[LANG]\{course\}} \\
        \texcommand{suget[LANG]\{program\}} \\
        \texcommand{suget[LANG]\{semester\}} \\
        \texcommand{suget\{supervisor\}} \\
    Textruta med tryckinformation längst ned till vänster på fram- och baksida.
    Verdana 8~punkter rak. Vänsterjusterad.
    % I Libreoffice: (Line spacing "fixed" 14~punkter.) Avstånd ovanför
    % 0~punkter, avstånd nedanför 0~punkter.
  } \\
  \bottomrule
\end{tabular}


\subsection{Sammanfattningssida}

Detta är sidan efter framsidan. Den innehåller uppsats titel, undertitel och
författarnamn följt av en sammanfattningen och nyckelorden, och därefter den
engelska sammanfattningen (eller 'abstract') och nyckelorden.

Om språket är svenska (specificerat med \verb|\usepackage[swedish]{babel}|) så
visas den svenska titeln, och den svenska sammanfattningen före det engelska
abstractet, i annat fall visas den engelska titeln och det engelska abstractet
före den svenska sammanfattningen.

\medskip

\begin{tabular}{p{.33\textwidth} p{.67\textwidth}}
  \toprule
  \thead{Typ} &
  \thead{Beskrivning} \\
  \midrule
%%% Wordmall: 'Uppsatstitel sidan2' (sans) -- \section*{}
  \thead{
    \Large\sffamily\mdseries
    Titel
  } &
  \makecell[{{p\linewidth}}]{
    %% Verdana 26~punkter. Vänsterjusterad. (Line spacing "at least"
    %% 15~punkter.) Avstånd ovanför 36~punkter, avstånd nedanför 24~punkter \\
    Uppsatstiteln (samma som '1 Rubrik').
  } \\
  \midrule
%%% Wordmall: 'Undertitel sidan2' (sans)
  \thead{
    \normalsize\sffamily\bfseries
    Undertitel
  } &
  \makecell[{{p\linewidth}}]{
    Används på sammanfattningssidan till undertitel och författarnamn på.
    Verdana 11~punkter, fetstil. Vänsterjusterad.
    % I Libreoffice: (Line spacing "at least" 15~punkter.)
    Avstånd ovanför 36~punkter, avstånd nedanför 18~punkter (samma som
    under-underrubrik men med större avstånd ovanför och under).
  } \\
  \midrule
%%% Wordmall: 'Sammanfattning' (sans) -- \section*{}
  \thead{
    \Large\sffamily\mdseries
    Rubrik
  } &
  \makecell[{{p\linewidth}}]{
    Onumrerad rubrik till sammanfattning/abstract
    %% Verdana 26~punkter. Vänsterjusterad. (Line spacing "at least"
    %% 15~punkter.) Avstånd ovanför 36~punkter, avstånd nedanför 24~punkter
    (samma som '1 Rubrik').
  } \\
  \midrule
%%% Wordmall: 'Nyckelord' (sans) samma som 'Heading 3'?)
  \thead{
    \normalsize\sffamily\textbf
    Nyckelordsrubrik
  } &
  \makecell[{{p\linewidth}}]{
    Onumrerad under-underrubrik. Verdana 11~punkter fetstil. Vänsterjusterad.
    % I Libreoffice (Line spacing "at least" 15~punkter.)
    Avstånd ovanför 24~punkter, avstånd nedanför 6~punkter (samma som
    under-underrubrik, men med dubbelt så stort avstånd ovanför).
  } \\
  \bottomrule
\end{tabular}


%%% Wordmall: 'Uppsatstitel' (sans, blå)
%% \textbf{Titel} -- Används bara på framsidan, till uppsatstiteln. Verdana
%% 18~punkter. Vänsterjusterad. (Line spacing "at least" 15~punkter.) Färg: blå.

%% %%% Wordmall: 'Undertitel' (sans, blå)
%% \textbf{Undertitel} -- Används bara på framsidan till undertitel och
%% författarnamn. Verdana 14~punkter, rak. Vänsterjusterad. (Line spacing "at
%% least" 24~punkter.) Avstånd ovanför 6~punkter, avstånd nedanför 28~punkter.
%% Färg: blå.

%% %%% Wordmall: 'Textruta' (sans, blå)
%% \textbf{Textruta} -- Används bara på fram- och baksida till tryckinformation
%% (textrutan längst ned till vänster). Verdana 8~punkter, rak.
%% Vänsterjusterad. (Line spacing "fixed" 14~punkter.) Avstånd ovanför 0~punkter,
%% avstånd nedanför 0~punkter. Färg: blå.

%% %%% Wordmall: 'Uppsatstitel sidan2' (sans)
%% \textbf{Titel sida 2} -- Används bara på sammanfattningssidan. Verdana
%% 26~punkter. Vänsterjusterad. (Line spacing "at least" 15~punkter.) Avstånd
%% ovanför 36~punkter, avstånd nedanför 24~punkter.

%% %%% Wordmall: 'Undertitel sidan2' (sans)
%% \textbf{Undertitel sida 2} -- Används bara på sammanfattningssidan till
%% undertitel och författarnamn på framsidan. Verdana 11~punkter, fetstil.
%% Vänsterjusterad. (Line spacing "at least" 15~punkter.) Avstånd ovanför
%% 36~punkter, avstånd nedanför 18~punkter.

%% %%% Wordmall: 'Heading 1' (sans)
%% \textbf{Rubrik (rubrik 1)} -- Numrerad kapitelrubrik.

%% %%% Wordmall: 'Heading 2' (sans)
%% \textbf{Underrubrik (rubrik 2)} -- Numrerad underrubrik.

%% %%% Wordmall: 'Heading 3' (sans)
%% \textbf{Under-underrubrik (rubrik 2)} -- Numrerad under-underrubrik.

%% %%% Wordmall: 'Sammanfattning' (sans)
%% \textbf{Sammanfattning}

%% %%% Wordmall: 'Nyckelord' (sans) samma som 'Heading 3'?)
%% \textbf{Nyckelordsrubrik}

%% %%% Wordmall: 'Innehållsförteckning' (sans) (rubrik)


%% \subsubsection{Rubriker}

%% Alla rubriker har stor bokstav i början, men i övrigt små bokstäver (utom i
%% namn, som ju alltid har stor bokstav i början oavsett om de kommer inuti en
%% rubrik eller ej). Rubriker avslutas inte med punkt.


%% \subsubsection{Underrubriker}


%% \subsection{Färg}

%% Hela dokumentet utom fram- och baksida har svart text på vit botten.

%% På fram- och baksida har texten Stockholms universitets blåa färg
%% (\texttt{\#002F5F}).


\subsection{Referenser}
\label{referenser}

Referenserna formateras automatiskt enligt APA-standard, baserat på den
BibTeX-information du angivit i din \verb|.bib|-fil.

Placera följande kommandon på den plats i din \verb|.tex|-fil där du vill ha
dina referenser. En onumrerad rubrik på ditt valda språk kommer automatiskt
infogas före referenserna (på svenska 'Referenser').

\begin{verbatim}
\nocite{*}                               % visa även oanvända referenser
\phantomsection                          % toc point to right page
\addcontentsline{toc}{section}{\refname} % inkludera i innehållsförteckning
\bibliography{dina_referenser.bib}       % ladda 'dina_referenser.bib'
\end{verbatim}

Om du använder \verb|\nocite{*}| så kommer referenskapitlet inkludera även de
referenser som du inte refererat till i din text, men dessa visas i grått för
att göra det lättare att se vilka det är.

\verb|\addcontentsline|-kommandot lägger till referenskapitlet i din uppsats
innehållsförteckning. Notera att detta kommando måste komma precis före
\verb|\bibliography|-kommandot för att sidreferensen i innehållsförteckningen
ska peka på rätt sida.

%% [eof]

\clearpage

%% -*- mode: latex; ispell-dictionary: "svenska" -*-

\section{Att använda \LaTeX}
\label{latex}

Detta är en uppsatsmall för \LaTeX\ för Stockholms universitet. Den kan
användas med \XeLaTeX\ på din egen dator eller med Overleaf
(\texttt{\href{https://overleaf.com/edu/su}{https://\linebreak[0]{}overleaf\linebreak[0]{}.com/\linebreak[0]{}edu/\linebreak[0]{}su}})
-- som student på Stockholms universitet har du automatiskt tillgång Overleaf.

Denna mall måste kompileras med \XeLaTeX, vilket du slå på manuellt i Overleaf.

\subsection{Overleaf}


\subsection{På egen dator}

\subsubsection{Installation av program}

Denna uppsatsmall kräver kommandot \texttt{xelatex} (det traditionella
kommandot \texttt{pdflatex} fungerar inte. \hl{FIXME: fungerar inte
pdflatex?}). Denna guide utgår också ifrån bekvämlighetsverktyget
\texttt{latexmk}.

\noindent Om du kör Linux (med Debian eller Ubuntu) kör du följande kommandon
(som användaren 'root') för att installera:

\begin{verbatim}
apt install texlive-xetex latexmk
\end{verbatim}

\noindent Dokumentationen till kommandot \texttt{latexmk} finns här:
\texttt{\href{https://mg.readthedocs.io/latexmk.html}{https://\linebreak[0]{}mg\linebreak[0]{}.readthedocs\linebreak[0]{}.io/\linebreak[0]{}latexmk\linebreak[0]{}.html}}


\subsubsection{Kompilering}

Det enklaste sättet att kompilera din uppsats är:

\begin{verbatim}
latexmk su-thesis.tex
\end{verbatim}

\noindent\texttt{latexmk} har inställningar i en fil som heter
\texttt{latexmkrc}. Om du installerar denna uppsatsmall i en underkatalog så
vill du förmodligen kopiera \texttt{latexmkrc} till katalogen där du har din
uppsats också.

Om du inte använder \texttt{latexmk} utan insisterar på att kompilera
själv så motsvarar detta följande kommandon (Ja! -- \texttt{xelatex} behöver
vanligtvis köras tre gånger för att vara säker på att både innehållsförteckning
och referenser stämmer överens med texten).

\begin{verbatim}
export TEXINPUTS=".//:$TEXINPUTS"
xelatex su-thesis.tex
bibtex su-thesis.bib
xelatex su-thesis.tex
xelatex su-thesis.tex
\end{verbatim}

\noindent Då och då råkar man skriva något galet i \LaTeX-koden, och kompileringen
stannar mitt i. Då får man upp en prompt, med ett frågetecken i början på en ny
rad. -- När detta händer skriv 'x' (och tryck enter) för att avsluta. Rätta
felet i din \texttt{.tex}-fil och kompilera om.


\subsection{Att skriva referenser}

För att skriva en referens i din text använder antingen kommandot
\texttt{\textbackslash{}citep\{\}} (för en parentetisk referens) eller
kommandot \texttt{\textbackslash{}citet\{\}} (för en referens utan parentes
runt författarnamnet).

\begin{table}[ht]
  \centering
  \renewcommand{\arraystretch}{1.25}%
  \begin{tabular}[t]{lcc}
    \toprule
    {\sffamily\textbf{Kommando}} &
    {\sffamily\textbf{Beskrivning}} &
    {\sffamily\textbf{Exempel}} \\
    \midrule
    \texttt{\textbackslash{}citep\{källa\}} &
    med parentes &
    \citep{bergman+wallin-2001} \\
    \texttt{\textbackslash{}citet\{källa\}} &
    utan parentes &
    \citet{bergman+wallin-2001} \\
    \midrule
    \texttt{\textbackslash{}citep\{källa1,källa2\}} &
    flera källor &
    \citep{bergman+wallin-2001,bergman-1977} \\
    \texttt{\textbackslash{}citet\{källa1,källa2\}} &
    flera källor &
    \citet{bergman+wallin-2001,bergman-1977} \\
    \midrule
    \texttt{\textbackslash{}citep[1--2]\{källa1\}} &
    sidnummer &
    \citep[1--2]{bergman-1977} \\
    \texttt{\textbackslash{}citep[jmf][]\{källa1\}} &
    text före &
    \citep[jmf][]{bergman-1977} \\
    \texttt{\textbackslash{}citep[jmf][1--2]\{källa1\}} &
    text + sidref. &
    \citep[jmf][1--2]{bergman-1977} \\
    \midrule
    \texttt{\textbackslash{}citealp\{källa\}} &
    som \texttt{\textbackslash{}citep\{källa\}} &
    \citealp{bergman+wallin-2001} \\
    \texttt{\textbackslash{}citealt\{källa\}} &
    som \texttt{\textbackslash{}citet\{källa\}} &
    \citealt{bergman+wallin-2001} \\
    \bottomrule
  \end{tabular}
  \caption{\LaTeX-kommandon för att skriva referenser (från \texttt{natbib}-paketet)}
\end{table}

\texttt{\textbackslash{}citealp} och \texttt{\textbackslash{}citealt}
kommandona fungerar som \texttt{\textbackslash{}citep} och
\texttt{\textbackslash{}citet} men utelämnar de omgivande parenteserna. Ibland
kan detta vara nödvändigt, som tex om man vill ange två källor med sidreferens
inom parentes. I detta fall kan man skriva
\texttt{(\textbackslash{}citealp[55]\{källa1\} och
  \textbackslash{}citealp[25]\{källa2\})}.

Det finns ytterligare kommandon som kan användas för att skapa referenser, hur
de används finns beskrivet i
\underline{\href{http://ftp.acc.umu.se/mirror/CTAN/macros/latex/contrib/natbib/natbib.pdf}{\texttt{natbib}-paketets referensmanual}}.

%% [eof]

\clearpage

%% -*- mode: latex; ispell-dictionary: "svenska" -*-

\section{Att använda mallen}
\label{mallen}


\subsection{Ladda mallen}
\label{variabler}

För att ladda mallen använd:

\begin{verbatim}
\usepackage{su-thesis}
\end{verbatim}


\subsection{Välja språk}
\label{sprak}

\noindent Språket på uppsatsen ställer man in med hjälp av paketet
\texttt{babel}. När du byter språk så kommer byts fram- och baksida ut mot
information på det relevanta språket (men resten av uppsatsen får du förstås
översätta själv). :)

\begin{verbatim}
\usepackage[swedish]{babel} % Swedish
%% \usepackage[UKenglish]{babel} % British English
%% \usepackage[USenglish]{babel} % American English
\end{verbatim}

\noindent\textbf{Notera:} När du bytt språk behöver du radera alla tempfiler
och kompilera om. Om du använder \texttt{latexmk} är detta enkelt.

\begin{verbatim}
latexmk -xelatex su-thesis.tex -C  % radera alla tempfiler
latexmk -xelatex su-thesis.tex     % kompilera om
\end{verbatim}


\subsection{Mallens variabler}
\label{variabler}

Mallen har ett antal variabler som man sätter med \verb|\suset[SPRÅK]{NAMN}|
(för tillfället kan bara \texttt{swedish} och \texttt{english} användas som
språk).

\textbf{Notera:} Du behöver ladda mallen innan du kan sätta variablerna.

\begin{verbatim}
%% Always used
\suset{author}{Förnamn Efternamn}
\suset{supervisor}{Förnamn1 Efternamn1[, Förnamn2 Efternamn2]}
\suset[swedish]{title}{Titel på svenska}
\suset[english]{title}{Title in English}
\suset[swedish]{abstract}{Sammanfattning på svenska}
\suset[english]{abstract}{Abstract in English}

%% Swedish metadata (only used if language is Swedish)
\suset[swedish]{subtitle}{Svensk undertitel}
\suset[swedish]{keywords}{nyckelord1, nyckelord2, nyckelord3, \ldots}
\suset[swedish]{department}{Institutionen för lingvistik}
\suset[swedish]{thesistype}{Examensarbete 15 hp}
\suset[swedish]{course}{Lingvistik -- kandidatkurs, LIN600}
\suset[swedish]{program}{Kandidatprogrammet i lingvistik 180 hp}
\suset[swedish]{semester}{Vårterminen 2020}

%% English metadata (only used if language is English)
\suset[english]{subtitle}{English Subtitle}
\suset[english]{keywords}{keyword1, keyword2, keyword3, \ldots}
\suset[english]{department}{Department of Linguistics}
\suset[english]{thesistype}{Bachelor's Thesis 15 ECTS credits}
\suset[english]{course}{Linguistics -- Bachelor's Course, LIN600}
\suset[english]{program}{Bachelor's Programme in Linguistics 180 ECTS credits}
\suset[english]{semester}{Spring semester 2020}
\end{verbatim}


\subsection{Fonetiska tecken (IPA)}
\label{ipa}

\hl{FIXME: Lägg till en \texttt{\textbackslash{}suipa\{...\}} funktion.}

Det finns en IPA-skrivmaskin på nätet som man kan använda för att skriva
IPA-symboler:
\texttt{\href{https://ipa.typeit.org/full/}{https://\linebreak[0]{}ipa\linebreak[0]{}.typeit\linebreak[0]{}.org/\linebreak[0]{}full/}}


\subsection{Ta bort inledande/avslutande sidor}
\label{genererade}

Mallen använder interna kommandon för att genera inledande och avslutande
sidor. För att ta bort en eller flera av dessa kan man definiera om de
relevanta kommandona så att det inte generar någon effekt. Detta görs med hjälp
av \LaTeX-kommandot \verb|\renewcommand|. Följande kommandon kan användas:

\begin{verbatim}
\renewcommand{\sumakefrontpage}{}     % ta bort framsida
\renewcommand{\sumakeabstractpage}{}  % ta bort sammanfattningssida
\renewcommand{\sutableofcontents}{}   % ta bort innehållsförteckning
\renewcommand{\sumakebackpage}{}      % ta bort baksida
\end{verbatim}

\noindent\textbf{Notera:} Raden med \verb|\renewcommand| måste komma efter
raden som laddar mallen.

%% [eof]

\clearpage

%% -*- mode: latex; ispell-dictionary: "svenska" -*-

\section{Strukturen på en uppsats}
\label{struktur}

\subsection{Rubriker}
\label{rubriker}

En uppsats vid Institutionen för Lingvistik innehåller oftast följande
huvudrubriker.

\begin{enumerate}
\item\nameref{rubrik.inledning}
\item\nameref{rubrik.bakgrund}
\item\nameref{rubrik.syfte}
\item\nameref{rubrik.metod}
\item\nameref{rubrik.resultat}
\item\nameref{rubrik.diskussion}
\item\nameref{rubrik.slutsats}
\item\nameref{rubrik.referenser} (onumrerad)
\end{enumerate}

\noindent Namnsättningen av underrubriker är däremot friare.

I förekommande fall kan det också tillkomma en eller fler bilagor.


\subsection{Inledning}
\label{rubrik.inledning}

Inledningen är oftast ungefär ¼--1 sida lång, och brukar inte innehålla några
underrubriker.

Här berättar du varför din forskningsfråga är viktig. – Inledningen kan
beskrivas som en tratt, i det att den börjar brett och sedan smalnar av till
att handla om just din undersökning, ditt ämne. Inledningen ger också läsaren
den bakgrund som behövs för att förstå ditt ämne.

Inledningen kan ta sitt avstamp i något dagsaktuellt, eller något större (som
till exempel ett samhällsproblem), men bör inte grundas i vad du personligen
tycker är intressant. Inledningen ska heller inte börja alltför långt ifrån
ämnet.


\subsection{Bakgrund}
\label{rubrik.bakgrund}

Brukar vara 2--9 sidor lång (vanligast är cirka 6 sidor) och har ofta
underrubriker specifika för ämnet. Bakgrunden kan ses som en sort förlängning
av inledningen som går in i mer detalj om rådande forskningsläge, och ger
läsaren den bakgrundskunskap och förankring som behövs för att kunna ta till
sig din undersökning.

Här presenteras utgångspunkten och det allmänna syftet. Vad är motivet och
ämnesvalet? Varför är just detta viktigt och intressant? Vad är bakgrunden till
det som skall studeras? Här kan man också gå igenom uppsatsens disposition och
huvuddrag. Vad handlar de olika kapitlen om? Viktiga termer och begrepp kan
också presenteras här.


\subsection{Syfte och frågeställningar}
\label{rubrik.syfte}

En mycket kort beskrivning av uppsatsens centrala frågeställningar. Vanligtvis
4–7 \emph{rader} lång (ett eller två korta stycken). Denna kan växa och bli
något längre särskilt om författaren väljer att ställa upp sina frågor i form
av en punktlista, men även i dessa fall har jag inte sett någon uppsats i
vilken denna del var längre än en halv sida.

Frågeställningarna är oftast formulerade som frågor, men kan också formuleras
som "syftet med studien är att beskriva..." eller liknande formuleringar.


\subsection{Metod och material}
\label{rubrik.metod}

1--8 sidor, men vanligast är 1--2 sidor. Om kapitlet är längre är det oftast
indelat i underrubriker, med en underrubrik vardera för de olika korpusar eller
datainsamlingsmetoder som används. Här beskrivs också ofta vilka avgränsningar
som gjorts, etiska aspekter vid intervju eller korpusanvändande, tekniska
begränsningar, hur intervju eller enkät sett ut, eller tänkbar skevhet eller
partiskhet i data eller metod.


\subsection{Resultat}
\label{rubrik.resultat}

5--30 sidor, vanligast cirka 15 sidor. Kapitlet är alltid indelat i
underkapitel, och oftast också med under-underrubriker.


\subsection{Diskussion}
\label{rubrik.diskussion}

1--7 sidor, men vanligast är 2--4 sidor. Har alltid underrubriker. Vanliga
underrubriker är "resultatdiskussion", "metoddiskussion", "etikdiskussion" och
"framtida forskning", men det än inte heller ovanligt med andra underrubriker
specifika för ämnet. Under-underrubriker är också vanliga.


\subsection{Slutsatser}
\label{rubrik.slutsats}

Överstiger oftast inte en sida, utom i längre uppsatser. Underrubriker
förekommer inte.


\subsection{Referenser}
\label{rubrik.referenser}

Ett onumrerat kapitel. Brukar vara 1--2 sidor (oftast ganska precis en sida) i
längd. Och innehåller bara en lista av de referenser refererats till i
uppsatsens övriga text. Om du i ditt \LaTeX dokument valt att visa alla
referenser (med kommandot \verb|\nocite{*}|) så ser uppsatsmallen till så att
de referenser du ännu inte refererat till i texten markeras med grå text
(istället för svart).

%% [eof]

\clearpage

\nocite{*} % show all references, even those not referred to in text
\bibliography{su-thesis-5-referenser.bib}
\clearpage

%% \begin{appendices}
%% \input{su-thesis-9-bilagor.tex}
%% \end{appendices}
%% \clearpage

\end{document}

%% [eof]

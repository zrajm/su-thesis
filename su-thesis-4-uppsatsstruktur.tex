%% -*- mode: latex; ispell-dictionary: "svenska" -*-

\section{Strukturen på en uppsats}
\label{struktur}

Nedan försöker jag beskriva praxis vid Institutionen för lingvistik på
Stockholms universitet, så som jag uppfattat den. Detta bygger delvis på
jämförelser mellan olika kandidatuppsatser, delvis på information från
\citet{schott+others-2007, wiren-2020} och \citet{uppsatsguide-2020}.

\textbf{Längd:} En typisk kandidatuppsats är 20--50 sidor och en magister-
eller masteruppsats är 30--60 sidor.

\begin{itemize}
\item1. \nameref{rubrik.inledning}
\item2. \nameref{rubrik.bakgrund}
\item3. \nameref{rubrik.syfte}
\item4. \nameref{rubrik.metod}
\item5. \nameref{rubrik.resultat}
\item6. \nameref{rubrik.diskussion}
\item7. \nameref{rubrik.slutsats}
\item\nameref{rubrik.referenser}
\item (+ Eventuella bilagor)
\end{itemize}

\noindent Namnsättningen på eventuella underrubriker är mestadels friare men
några underrubriker återkommer dock oftare än andra, och i förekommande fall
finns detta beskrivet nedan.

Exakt vilka rubriker som används och hur de är underordnade varandra varierar
från uppsatsguide till uppsatsguide (ofta inkluderas tex rubrikerna
\emph{\nameref{rubrik.bakgrund}} och \emph{\nameref{rubrik.syfte}} under
\emph{\nameref{rubrik.inledning}}). Det som beskrivs nedan är den vanligaste
uppsättningen rubriker för uppsatser vid Institutionen för Lingvistik, på
Stockholms universitet.


\subsection{Inledning}
\label{rubrik.inledning}

Inledningen är oftast ungefär ¼--1 sida lång, och brukar inte innehålla några
underrubriker.

Här berättar du varför din forskningsfråga är viktig. – Inledningen kan
beskrivas som en tratt, i det att den börjar brett och sedan smalnar av till
att handla om just din undersökning, ditt ämne.

Inledningen kan ta sitt avstamp i något dagsaktuellt, eller något större (som
till exempel ett samhällsproblem), men bör inte grundas i vad du personligen
tycker är intressant. Inledningen ska heller inte börja alltför långt ifrån
ämnet.


\subsection{Bakgrund}
\label{rubrik.bakgrund}

Brukar vara 2--9 sidor lång (vanligast är cirka 6 sidor) och har ofta
underrubriker specifika för ämnet. Bakgrunden kan ses som en sort förlängning
av inledningen som går in i mer detalj om rådande forskningsläge, och ger
läsaren den bakgrundskunskap och förankring som behövs för att kunna ta till
sig din undersökning.

Syftet med bakgrundskapitlet är att peka ut ett problemområde, beskriva
tidigare forskning och det aktuella forskningsläget inom området (med
åtföljande referenser), samt identifiera en kunskapslucka som utgör motivering
till varför undersökningen ska göras.

Här presenteras utgångspunkten och det allmänna syftet. Vad är motivet och
ämnesvalet? Varför är just detta viktigt och intressant? Vad är bakgrunden till
det som skall studeras? Här kan man också gå igenom uppsatsens disposition och
huvuddrag. Vad handlar de olika kapitlen om? Viktiga termer och begrepp kan
också presenteras här. Tänk dock på att inte ha inte mer bakgrundsinformation
än nödvändigt. En onödigt lång bakgrund gör inte din text bättre.


\subsection{Syfte och frågeställningar}
\label{rubrik.syfte}

En mycket kort beskrivning av uppsatsens centrala frågeställningar. Vanligtvis
4–7 \emph{rader} lång (ett eller två korta stycken). Denna kan växa och bli
något längre särskilt om författaren väljer att ställa upp sina frågor i form
av en punktlista, men även i dessa fall har jag inte sett någon uppsats i
vilken denna del var längre än en halv sida.

Frågeställningarna är oftast formulerade som frågor, men kan också formuleras
som "syftet med studien är att beskriva..." eller liknande formuleringar. Syfte
och frågeställningar är den mest centrala delen i uppsatsen, och styr mycket
hur de övriga delarna kommer se ut.


\subsection{Metod och material}
\label{rubrik.metod}

1--8 sidor, men vanligast är 1--2 sidor. Beskriver hur studien gjorts, och
motiverar varför du gjort som du gjort.

Om kapitlet är längre är det oftast indelat i underrubriker, med en underrubrik
vardera för de olika korpusar eller datainsamlingsmetoder som används. Här
beskrivs också ofta vilka avgränsningar som gjorts, etiska aspekter vid
intervju eller korpusanvändande, tekniska begränsningar, hur intervju eller
enkät sett ut, eller tänkbar skevhet eller partiskhet i data eller metod.


\subsection{Resultat}
\label{rubrik.resultat}

5--30 sidor, vanligast cirka 15 sidor. Redovisar vad du kommit fram till och
redovisar det resultat som är relevant för syftet med studien. Kapitlet är
alltid indelat i underkapitel, och oftast också med under-underrubriker.


\subsection{Diskussion}
\label{rubrik.diskussion}

1--7 sidor, men vanligast är 2--4 sidor. Diskuterar hur man kan se på
resultatet utifrån olika synvinklar och kopplar resultatet till den tidigare
forskningen. Har alltid underrubriker. Vanliga underrubriker är
"resultatdiskussion", "metoddiskussion", "etikdiskussion" och "framtida
forskning", men det än inte heller ovanligt med andra underrubriker specifika
för ämnet. Under-underrubriker är också vanliga.


\subsection{Slutsatser}
\label{rubrik.slutsats}

Överstiger oftast inte en sida, utom i längre uppsatser. Underrubriker
förekommer inte.


\subsection{Referenser}
\label{rubrik.referenser}

Ett onumrerat kapitel. Brukar vara 1--2 sidor (oftast ganska precis en sida) i
längd. Och innehåller bara en lista av de referenser refererats till i
uppsatsens övriga text. Om du i ditt \LaTeX dokument valt att visa alla
referenser (med kommandot \verb|\nocite{*}|) så ser uppsatsmallen till så att
de referenser du ännu inte refererat till i texten markeras med grå text
(istället för svart).

%% [eof]

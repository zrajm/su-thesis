%% -*- mode: latex; ispell-dictionary: "svenska" -*-

\section{Stockholms universitets uppsatsmall}
\label{utseende}

Denna uppsatsmall bygger delvis på Stockholms universitets studentuppsatsmall
för Word, delvis på andra stilrekommendationer från Stockholms universitet som
jag kunnat hitta. Jag har också undersökt och jämfört ett tiotal
kandidatuppsatser skrivna vid institutionen för lingvistik och därifrån hämtat
inspiration till sådant som jag inte hittat beskrivet.

\medskip

\begin{itemize}
\raggedright
\item Stockholms universitetsbiblioteks studentuppsatsmall (finns bara till
  Word):\\
  \url{https://su.se/biblioteket/guider/guider/studentuppsatsmall}

\item Stockholms universitetsbiblioteks "Mallar och visuell identitet för
  avhandling" \\
  (Word-mallar för olika typer av avhandlingar): \\
  \url{http://su.se/biblioteket/avhandlingsmallar}

\item Stockholms universitetsbiblioteks "Dokumentmall i Word för
  doktorsavhandlingar" (PDF): \\
  \url{https://su.se/polopoly\_fs/1.264229.1576661286!/menu/standard/
    file/Instruktioner%20SU\_Wordmall\_20180509.pdf}

\item Institutionen för svenska och flerspråkighets "Uppsatsmall och
  Word-tips": \\
  \url{https://su.se/svefler/publikationer/uppsatsarkiv/
    uppsatsmall-och-word-tips}

\item Stockholms universitets grafiska manual (allmän information om hur
  grafisk profil, ingenting specifikt om utseende på uppsatser/avhandlingar):
  \\
  \url{https://su.se/medarbetare/kommunikation/grafisk-manual}
\end{itemize}

%% \subsection{Typsnitt}

%% Brödtexten är satt i Times New Roman, medan titel, rubrik, underrubriker etc.
%% är satta i Verdana. På fram- och baksida är all text och universitetets logotyp
%% satta i Stockholms universitetets blåa färg, på övriga sidor är all text svart.

%% %% Brödtext (12 uppsatser):
%% %%  Blankrad: 11 / Indrag: 1 (Bark 2018)
%% %%  Vänsterjusterad: 7, Spärrad: 5 (Bäckström 2015, Lyxell 2016, Petersdotter
%% %%                                 2018, Bark 2018, Klintberg 2018)

%% \subsubsection{Times New Roman}

%% Brödtexten används för den huvudsakliga texten i uppsatsen. Den används också
%% för texten i styckena för sammanfattning/abstract, och nyckelord/keywords. Den
%% används däremot \emph{inte} för innehållsförteckning eller i textrutorna på
%% fram- och baksida.

%% %\textbf{Styckeindrag} -- Inget.

\subsection{Textstilar}

För att \emph{kursivera text} använd \texcommand{emph}. För \textbf{fetstil}
använd \texcommand{textbf}. Om du vill skriva med \texttt{skriv\-maskins\-stil}
(vilket ofta används för variabelnamn eller kod i text om programmering) använd
\texcommand{texttt}. Du kan också \underline{stryka under text} med
\texcommand{underline}. Det finns många andra \LaTeX{}-stilkommandon du också
kan använda men dessa är de vanligaste.

\LaTeX{} är oftast rättså bra på att luska ut hur ord ska avstavas men ibland
kan det gå snett och ett ord kan sticka långt ut i marginalen. För att fixa
detta när det händer det kan du antingen skriva \verb|\-| inuti ordet där du
vill tillåta avstavning (dvs \verb|Fo\-ne\-tik\-lab\-bet|). Alternativt så kan
du lägga till ett skapa en lista i början av ditt dokument (den måste komma
före \verb|\begin{document}|) med alla avstavningsmönster du behöver.

\begin{verbatim}
\hyphenation{Fo-ne-tik-lab-bet}
\hyphenation{kort-films-festi-vals-med-ar-betar-fest}
\end{verbatim}

En fördel med att lägga till avstavningsmönster i början av dokumentet är att
om ett ord förekommer flera gånger så behöver du bara beskriva avstavningen på
ett ställe. En annan fördel är att du kan söka på ordet i din text och inte
måste komma ihåg att just det ordet var avstavat.

För att skriva webbadresser i din text använd kommandot \texcommand{url}. Det
ser ut såhär i text \url{http://www.exempel.com/}, skapar en länk i texten och
hanterar eventuella radbrytningar på ett bra sätt (utan att stoppa in
bindestreck som förändrar vart adressen pekar).


\subsection{Rubriker och brödtext}

Sidmarginalerna är 2,5~cm på alla fyra sidor, utom upptill där den är 2,725~cm.
Rubriker är vänsterjusterade och satta med Verdana, medan brödtext har rak
högermarginal och är satt med Times New Roman. Stycken är separerade med ett
vertikalt mellanrum utan styckeindrag.

\medskip

\begin{tabular}{p{.33\textwidth} p{.67\textwidth}}
  \toprule
  \thead{Typ} &
  \thead{Beskrivning} \\
  \midrule
%%% Wordmall: 'Heading 1' (sans)
  \thead{
    \Large\sffamily\mdseries
    1~Rubrik
  } &
  \makecell[{{p\linewidth}}]{
    \texcommand{section} \\
    Numrerad kapitelrubrik. Verdana 26~punkter. Vänsterjusterad.
    % I Libreoffice: (Line spacing "at least" 15~punkter.)
    Avstånd ovanför 36~punkter, avstånd nedanför 24~punkter.
  } \\
  \midrule
%%% Wordmall: 'Heading 2' (sans) -- \subsection{}
  \thead{
    \large\sffamily\mdseries
    1.1~Underrubrik
  } &
  \makecell[{{p\linewidth}}]{
    \texcommand{subsection} \\
    Numrerad underrubrik. Verdana 16~punkter fetstil. Vänsterjusterad.
    % I Libreoffice: (Line spacing "at least" 15~punkter.)
    Avstånd ovanför 24~punkter, avstånd nedanför 6~punkter.
  } \\
  \midrule
%%% Wordmall: 'Heading 3' (sans) -- \subsubsection{}
  \thead{
    \normalsize\sffamily\bfseries
    1.1.1~Under-underrubrik
  } &
  \makecell[{{p\linewidth}}]{
    \texcommand{subsubsection} \\
    Numrerad under-underrubrik. Verdana 11~punkter fetstil. Vänsterjusterad.
    % I Libreoffice: (Line spacing "at least" 15~punkter.)
    Avstånd ovanför 12~punkter, avstånd nedanför 6~punkter.
  } \\
  \midrule
%%% Wordmall: 'Default Style'
  %% Set with:
  %%   \documentclass[11pt]{article} % set base font size
  %%   \setlength{\parindent}{0cm}   % remove 1st line of paragraph indentation
  %%   \setlength{\parskip}{6pt}     % add vertical space between paragraphs
  \thead{
    \normalsize\rmfamily\mdseries
    Brödtext
  } &
  \makecell[{{p\linewidth}}]{
    Times New Roman 11~punkter. Vänsterjusterad.
    % I Libreoffice: (Line spacing: Single.)
    Avstånd ovanför 0~punkter, avstånd nedanför 6~punkter.
  } \\
  \bottomrule
%% %%% Wordmall: 'Header'
%%   Sidhuvud &
%%   \hl{FIXME}
%%   \\
%%   \midrule
%% %%% Wordmall: 'Footer'
%%   Sidfot &
%%   \hl{FIXME}
%%   \\
%%   \midrule
\end{tabular}


\subsection{Fram- och baksida}

Texten på fram- och baksidor är helt och hållet satt med Verdana (dvs
sansseriff) och har Stockholms universitets blå färg (på både text och
logotyp).

\medskip

\begin{tabular}{p{.33\textwidth} p{.67\textwidth}}
  \toprule
  \thead{Typ} &
  \thead{Beskrivning} \\
  \midrule
%%% Wordmall: 'Uppsatstitel' (sans, blå)
  \thead{
    \color{sublue}\LARGE\sffamily\mdseries
    Titel
  } &
  \makecell[{{p\linewidth}}]{
    \texcommand{suset[LANG]\{title\}} \\
    Uppsatstiteln på framsidan. Verdana 28~punkter rak. Vänsterjusterad.
    % I Libreoffice: (Line spacing "at least" 15~punkter.)
  } \\
  \midrule
%%% Wordmall: 'Undertitel' (sans, blå)
  \thead{
    \color{sublue}\subtitlesize\sffamily\mdseries
    Undertitel
  } &
  \makecell[{{p\linewidth}}]{
    \texcommand{suset[LANG]\{subtitle\}} \\
    Undertitel och författarnamn på framsidan. Verdana 14~punkter rak.
    Vänsterjusterad.
    % I Lireoffice: (Line spacing "at least" 24~punkter.) Avstånd ovanför
    % 6~punkter, avstånd nedanför 28~punkter.
  } \\
  \midrule
%%% Wordmall: 'Textruta' (sans, blå)
  \thead{
    \color{sublue}\small\sffamily\mdseries
    Textruta
  } &
  \makecell[{{p\linewidth}}]{
        \texcommand{suget[LANG]\{department\}} \\
        \texcommand{suget[LANG]\{thesistype\}} \\
        \texcommand{suget[LANG]\{course\}} \\
        \texcommand{suget[LANG]\{program\}} \\
        \texcommand{suget[LANG]\{semester\}} \\
        \texcommand{suget\{supervisor\}} \\
    Textruta med tryckinformation längst ned till vänster på fram- och baksida.
    Verdana 8~punkter rak. Vänsterjusterad.
    % I Libreoffice: (Line spacing "fixed" 14~punkter.) Avstånd ovanför
    % 0~punkter, avstånd nedanför 0~punkter.
  } \\
  \bottomrule
\end{tabular}


\subsection{Sammanfattningssida}

Detta är sidan efter framsidan. Den innehåller uppsats titel, undertitel och
författarnamn följt av en sammanfattningen och nyckelorden, och därefter den
engelska sammanfattningen (eller 'abstract') och nyckelorden.

Om språket är svenska (specificerat med \verb|\usepackage[swedish]{babel}|) så
visas den svenska titeln, och den svenska sammanfattningen före det engelska
abstractet, i annat fall visas den engelska titeln och det engelska abstractet
före den svenska sammanfattningen.

\medskip

\begin{tabular}{p{.33\textwidth} p{.67\textwidth}}
  \toprule
  \thead{Typ} &
  \thead{Beskrivning} \\
  \midrule
%%% Wordmall: 'Uppsatstitel sidan2' (sans) -- \section*{}
  \thead{
    \Large\sffamily\mdseries
    Titel
  } &
  \makecell[{{p\linewidth}}]{
    %% Verdana 26~punkter. Vänsterjusterad. (Line spacing "at least"
    %% 15~punkter.) Avstånd ovanför 36~punkter, avstånd nedanför 24~punkter \\
    Uppsatstiteln (samma som '1 Rubrik').
  } \\
  \midrule
%%% Wordmall: 'Undertitel sidan2' (sans)
  \thead{
    \normalsize\sffamily\bfseries
    Undertitel
  } &
  \makecell[{{p\linewidth}}]{
    Används på sammanfattningssidan till undertitel och författarnamn på.
    Verdana 11~punkter, fetstil. Vänsterjusterad.
    % I Libreoffice: (Line spacing "at least" 15~punkter.)
    Avstånd ovanför 36~punkter, avstånd nedanför 18~punkter (samma som
    under-underrubrik men med större avstånd ovanför och under).
  } \\
  \midrule
%%% Wordmall: 'Sammanfattning' (sans) -- \section*{}
  \thead{
    \Large\sffamily\mdseries
    Rubrik
  } &
  \makecell[{{p\linewidth}}]{
    Onumrerad rubrik till sammanfattning/abstract
    %% Verdana 26~punkter. Vänsterjusterad. (Line spacing "at least"
    %% 15~punkter.) Avstånd ovanför 36~punkter, avstånd nedanför 24~punkter
    (samma som '1 Rubrik').
  } \\
  \midrule
%%% Wordmall: 'Nyckelord' (sans) samma som 'Heading 3'?)
  \thead{
    \normalsize\sffamily\textbf
    Nyckelordsrubrik
  } &
  \makecell[{{p\linewidth}}]{
    Onumrerad under-underrubrik. Verdana 11~punkter fetstil. Vänsterjusterad.
    % I Libreoffice (Line spacing "at least" 15~punkter.)
    Avstånd ovanför 24~punkter, avstånd nedanför 6~punkter (samma som
    under-underrubrik, men med dubbelt så stort avstånd ovanför).
  } \\
  \bottomrule
\end{tabular}


%%% Wordmall: 'Uppsatstitel' (sans, blå)
%% \textbf{Titel} -- Används bara på framsidan, till uppsatstiteln. Verdana
%% 18~punkter. Vänsterjusterad. (Line spacing "at least" 15~punkter.) Färg: blå.

%% %%% Wordmall: 'Undertitel' (sans, blå)
%% \textbf{Undertitel} -- Används bara på framsidan till undertitel och
%% författarnamn. Verdana 14~punkter, rak. Vänsterjusterad. (Line spacing "at
%% least" 24~punkter.) Avstånd ovanför 6~punkter, avstånd nedanför 28~punkter.
%% Färg: blå.

%% %%% Wordmall: 'Textruta' (sans, blå)
%% \textbf{Textruta} -- Används bara på fram- och baksida till tryckinformation
%% (textrutan längst ned till vänster). Verdana 8~punkter, rak.
%% Vänsterjusterad. (Line spacing "fixed" 14~punkter.) Avstånd ovanför 0~punkter,
%% avstånd nedanför 0~punkter. Färg: blå.

%% %%% Wordmall: 'Uppsatstitel sidan2' (sans)
%% \textbf{Titel sida 2} -- Används bara på sammanfattningssidan. Verdana
%% 26~punkter. Vänsterjusterad. (Line spacing "at least" 15~punkter.) Avstånd
%% ovanför 36~punkter, avstånd nedanför 24~punkter.

%% %%% Wordmall: 'Undertitel sidan2' (sans)
%% \textbf{Undertitel sida 2} -- Används bara på sammanfattningssidan till
%% undertitel och författarnamn på framsidan. Verdana 11~punkter, fetstil.
%% Vänsterjusterad. (Line spacing "at least" 15~punkter.) Avstånd ovanför
%% 36~punkter, avstånd nedanför 18~punkter.

%% %%% Wordmall: 'Heading 1' (sans)
%% \textbf{Rubrik (rubrik 1)} -- Numrerad kapitelrubrik.

%% %%% Wordmall: 'Heading 2' (sans)
%% \textbf{Underrubrik (rubrik 2)} -- Numrerad underrubrik.

%% %%% Wordmall: 'Heading 3' (sans)
%% \textbf{Under-underrubrik (rubrik 2)} -- Numrerad under-underrubrik.

%% %%% Wordmall: 'Sammanfattning' (sans)
%% \textbf{Sammanfattning}

%% %%% Wordmall: 'Nyckelord' (sans) samma som 'Heading 3'?)
%% \textbf{Nyckelordsrubrik}

%% %%% Wordmall: 'Innehållsförteckning' (sans) (rubrik)


%% \subsubsection{Rubriker}

%% Alla rubriker har stor bokstav i början, men i övrigt små bokstäver (utom i
%% namn, som ju alltid har stor bokstav i början oavsett om de kommer inuti en
%% rubrik eller ej). Rubriker avslutas inte med punkt.


%% \subsubsection{Underrubriker}


%% \subsection{Färg}

%% Hela dokumentet utom fram- och baksida har svart text på vit botten.

%% På fram- och baksida har texten Stockholms universitets blåa färg
%% (\texttt{\#002F5F}).


\subsection{Referenser}
\label{referenser}

Referenserna formateras automatiskt enligt APA-standard, baserat på den
BibTeX-information du angivit i din \verb|.bib|-fil.

Placera följande kommandon på den plats i din \verb|.tex|-fil där du vill ha
dina referenser. En onumrerad rubrik på ditt valda språk kommer automatiskt
infogas före referenserna (på svenska 'Referenser').

\begin{verbatim}
\nocite{*}                               % visa även oanvända referenser
\phantomsection                          % toc point to right page
\addcontentsline{toc}{section}{\refname} % inkludera i innehållsförteckning
\bibliography{dina_referenser.bib}       % ladda 'dina_referenser.bib'
\end{verbatim}

Om du använder \verb|\nocite{*}| så kommer referenskapitlet inkludera även de
referenser som du inte refererat till i din text, men dessa visas i grått för
att göra det lättare att se vilka det är.

\verb|\addcontentsline|-kommandot lägger till referenskapitlet i din uppsats
innehållsförteckning. Notera att detta kommando måste komma precis före
\verb|\bibliography|-kommandot för att sidreferensen i innehållsförteckningen
ska peka på rätt sida.

%% [eof]

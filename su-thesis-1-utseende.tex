%% -*- mode: latex; ispell-dictionary: "svenska" -*-

\section{Stockholms universitets uppsatsmall}
\label{utseende}

Stockholms Universitet har en uppsatsmall, men dessvärre finns den bara till
\textit{Word}. Den finns här: \texttt{\href{https://www.su.se/biblioteket/guider/guider/studentuppsatsmall}{https://\linebreak[0]{}www.su.se/\linebreak[0]{}biblioteket/\linebreak[0]{}guider/\linebreak[0]{}guider/\linebreak[0]{}studentuppsatsmall}}

Ytterligare information om Stockholms universitets grafiska profil kan hittas här:
\texttt{\href{https://www.su.se/medarbetare/kommunikation/grafisk-manual}{https://\linebreak[0]{}www.su.se/medarbetare/\linebreak[0]{}kommunikation/\linebreak[0]{}grafisk-manual}}.

"Mallar och visuell identitet för avhandling"
\verb|http://su.se/biblioteket/avhandlingsmallar|


PDF "Dokumentmall i Word för doktorsavhandlingar"
\verb|https://www.su.se/polopoly_fs/1.264229.1576661286!/menu/standard/file/Instruktioner%20SU_Wordmall_20180509.pdf|


\subsection{Sidmarginaler}

Sidmarginalerna är 2,5 cm på alla sidor.


\subsection{Sidnumrering}


%% \subsection{Typsnitt}

%% Brödtexten är satt i Times New Roman, medan titel, rubrik, underrubriker etc.
%% är satta i Verdana. På fram- och baksida är all text och universitetets logotyp
%% satta i Stockholms universitetets blåa färg, på övriga sidor är all text svart.

%% %% Brödtext (12 uppsatser):
%% %%  Blankrad: 11 / Indrag: 1 (Bark 2018)
%% %%  Vänsterjusterad: 7, Spärrad: 5 (Bäckström 2015, Lyxell 2016, Petersdotter
%% %%                                 2018, Bark 2018, Klintberg 2018)

%% \subsubsection{Times New Roman}

%% Brödtexten används för den huvudsakliga texten i uppsatsen. Den används också
%% för texten i styckena för sammanfattning/abstract, och nyckelord/keywords. Den
%% används däremot \emph{inte} för innehållsförteckning eller i textrutorna på
%% fram- och baksida.

%% %\textbf{Styckeindrag} -- Inget.


\subsection{Stilar}

\begin{longtable}{| p{.40\textwidth} | p{.60\textwidth} |}
  \hline
%%% Wordmall: 'Default Style'
  %% Set with:
  %%   \documentclass[11pt]{article} % set base font size
  %%   \setlength{\parindent}{0cm}   % remove 1st line of paragraph indentation
  %%   \setlength{\parskip}{6pt}     % add vertical space between paragraphs
  Brödtext &
  Times New Roman 11~punkter. Vänsterjusterad.
  % I Libreoffice: (Line spacing: Single.)
  Avstånd ovanför 0~punkter, avstånd nedanför 6~punkter.
  \\\hline
%% %%% Wordmall: 'Header'
%%   Sidhuvud &
%%   \hl{FIXME}
%%   \\\hline
%% %%% Wordmall: 'Footer'
%%   Sidfot &
%%   \hl{FIXME}
%%   \\\hline
\end{longtable}


\subsubsection{Fram- och baksida}

På fram- och baksida är all text satt i Verdana, och allting (text och logotyp)
har Stockholms universitets blå färg.

\begin{longtable}{| p{.40\textwidth} | p{.60\textwidth} |}
  \hline
%%% Wordmall: 'Uppsatstitel' (sans, blå)
  {\sffamily\raggedright\LARGE\color{sublue}Titel} &
  Uppsatstiteln på framsidan. Verdana 28~punkter rak. Vänsterjusterad.
  % I Libreoffice: (Line spacing "at least" 15~punkter.)
  \\\hline
%%% Wordmall: 'Undertitel' (sans, blå)
  {\color{sublue}\sffamily\large{}Undertitel} &
  Undertitel och författarnamn på framsidan. Verdana 14~punkter rak.
  Vänsterjusterad.
  % I Lireoffice: (Line spacing "at least" 24~punkter.) Avstånd ovanför
  % 6~punkter, avstånd nedanför 28~punkter.
  \\\hline
%%% Wordmall: 'Textruta' (sans, blå)
  {\color{sublue}\sffamily\small{}Textruta} &
  Textruta med tryckinformation längst ned till vänster på fram- och baksida.
  Verdana 8~punkter rak. Vänsterjusterad.
  % I Libreoffice: (Line spacing "fixed" 14~punkter.) Avstånd ovanför
  % 0~punkter, avstånd nedanför 0~punkter.
  \\\hline
\end{longtable}


\subsubsection{Sammanfattningssida / Abstract}

\begin{longtable}{| p{.40\textwidth} | p{.60\textwidth} |}
  \hline
%%% Wordmall: 'Uppsatstitel sidan2' (sans) -- \section*{}
  {\sffamily\raggedright\Large{}Titel sidan 2} &
  %% Verdana 26~punkter. Vänsterjusterad. (Line spacing "at least"
  %% 15~punkter.) Avstånd ovanför 36~punkter, avstånd nedanför 24~punkter \\
  Uppsatstiteln på sammanfattningssidan. Samma stil som '1 Rubrik'.
  \\\hline
%%% Wordmall: 'Undertitel sidan2' (sans)
  {\sffamily\bfseries{}Undertitel sidan 2} &
  Används bara på sammanfattningssidan till undertitel och författarnamn på.
  Verdana 11~punkter, fetstil. Vänsterjusterad.
  % I Libreoffice: (Line spacing "at least" 15~punkter.)
  Avstånd ovanför 36~punkter, avstånd nedanför 18~punkter (Typsnittstorlek är
  densamma som i under-underrubrik men här är avståndet ovan/under större).
  \\\hline
%%% Wordmall: 'Sammanfattning' (sans) -- \section*{}
  {\sffamily\Large\raggedright{}Rubrik sammanfattning} &
  Onumrerad rubrik till sammanfattning/abstract.
  %% Verdana 26~punkter. Vänsterjusterad. (Line spacing "at least"
  %% 15~punkter.) Avstånd ovanför 36~punkter, avstånd nedanför 24~punkter
  (samma som '1 Rubrik').
  \\\hline
%%% Wordmall: 'Nyckelord' (sans) samma som 'Heading 3'?)
  {\sffamily\bfseries{}Nyckelordsrubrik} &
  Onumrerad under-underrubrik. Verdana 11~punkter fetstil. Vänsterjusterad.
  % I Libreoffice (Line spacing "at least" 15~punkter.)
  Avstånd ovanför 24~punkter, avstånd nedanför 6~punkter. (Samma som
  under-underrubrik, men här är avståndet ovanför dubbelt så stort).
  \\\hline
\end{longtable}


\subsubsection{Inuti uppsatsen}

Inuti uppsatsen används sansserifftypsnitt för alla rubriker, samt för texten i
innehållsförteckningen. All text är svart.

\begin{longtable}{| p{.40\textwidth} | p{.60\textwidth} |}
  \hline
%%% Wordmall: 'Innehållsförteckning' (sans) (rubrik) -- \section*{}
  {\sffamily\raggedright\Large{}Rubrik innehåll} &
  Onumrerad rubrik till innehållsförteckning.
  %% Verdana 26~punkter. Vänsterjusterad. (Line spacing "at least"
  %% 15~punkter.) Avstånd ovanför 36~punkter, avstånd nedanför 24~punkter
  (samma som '1 Rubrik').
  \\\hline
%%% Wordmall: FIXME
  {\sffamily\textbf{Kapitelrubrik i innehållsförteckning}} &
  Text i innehållsförteckning. Verdana 11~punkter fetstil. Vänsterjusterad.
  %I Libreoffice: (Line spacing: Single.)
  Avstånd ovanför 0~punkter, avstånd nedanför 5~punkter.
  \\\hline
%%% Wordmall: FIXME
  {\sffamily{}Underrubrik i innehållsförteckning} &
  Text i innehållsförteckning. Verdana 11~punkter rak. Vänsterjusterad.
  % I Libreoffice: (Line spacing: Single.)
  Avstånd ovanför 0~punkter, avstånd nedanför 5~punkter.
  \\\hline
%%% Wordmall: 'Heading 1' (sans)
  {\sffamily\Huge{}1 Rubrik} &
  Numrerad kapitelrubrik. Verdana 26~punkter. Vänsterjusterad.
  % I Libreoffice: (Line spacing "at least" 15~punkter.)
  Avstånd ovanför 36~punkter, avstånd nedanför 24~punkter.
  \\\hline
%%% Wordmall: 'Heading 2' (sans) -- \subsection{}
  {\sffamily\raggedright\bfseries\large{}1.1 Underrubrik} &
  Numrerad underrubrik. Verdana 16~punkter fetstil. Vänsterjusterad.
  % I Libreoffice: (Line spacing "at least" 15~punkter.)
  Avstånd ovanför 24~punkter, avstånd nedanför 6~punkter.
  \\\hline
%%% Wordmall: 'Heading 3' (sans) -- \subsubsection{}
  {\sffamily\raggedright\bfseries\normalsize{}1.1.1 Under-underrubrik} &
  Numrerad under-underrubrik. Verdana 11~punkter fetstil. Vänsterjusterad.
  % I Libreoffice: (Line spacing "at least" 15~punkter.)
  Avstånd ovanför 12~punkter, avstånd nedanför 6~punkter. (Samma som i
  undertiteln på sidan 2 men här är avståndet ovan/under mindre).
  \\\hline
\end{longtable}


%%% Wordmall: 'Uppsatstitel' (sans, blå)
%% \textbf{Titel} -- Används bara på framsidan, till uppsatstiteln. Verdana
%% 18~punkter. Vänsterjusterad. (Line spacing "at least" 15~punkter.) Färg: blå.

%% %%% Wordmall: 'Undertitel' (sans, blå)
%% \textbf{Undertitel} -- Används bara på framsidan till undertitel och
%% författarnamn. Verdana 14~punkter, rak. Vänsterjusterad. (Line spacing "at
%% least" 24~punkter.) Avstånd ovanför 6~punkter, avstånd nedanför 28~punkter.
%% Färg: blå.

%% %%% Wordmall: 'Textruta' (sans, blå)
%% \textbf{Textruta} -- Används bara på fram- och baksida till tryckinformation
%% (textrutan längst ned till vänster). Verdana 8~punkter, rak.
%% Vänsterjusterad. (Line spacing "fixed" 14~punkter.) Avstånd ovanför 0~punkter,
%% avstånd nedanför 0~punkter. Färg: blå.

%% %%% Wordmall: 'Uppsatstitel sidan2' (sans)
%% \textbf{Titel sida 2} -- Används bara på sammanfattningssidan. Verdana
%% 26~punkter. Vänsterjusterad. (Line spacing "at least" 15~punkter.) Avstånd
%% ovanför 36~punkter, avstånd nedanför 24~punkter.

%% %%% Wordmall: 'Undertitel sidan2' (sans)
%% \textbf{Undertitel sida 2} -- Används bara på sammanfattningssidan till
%% undertitel och författarnamn på framsidan. Verdana 11~punkter, fetstil.
%% Vänsterjusterad. (Line spacing "at least" 15~punkter.) Avstånd ovanför
%% 36~punkter, avstånd nedanför 18~punkter.

%% %%% Wordmall: 'Heading 1' (sans)
%% \textbf{Rubrik (rubrik 1)} -- Numrerad kapitelrubrik.

%% %%% Wordmall: 'Heading 2' (sans)
%% \textbf{Underrubrik (rubrik 2)} -- Numrerad underrubrik.

%% %%% Wordmall: 'Heading 3' (sans)
%% \textbf{Under-underrubrik (rubrik 2)} -- Numrerad under-underrubrik.

%% %%% Wordmall: 'Sammanfattning' (sans)
%% \textbf{Sammanfattning}

%% %%% Wordmall: 'Nyckelord' (sans) samma som 'Heading 3'?)
%% \textbf{Nyckelordsrubrik}

%% %%% Wordmall: 'Innehållsförteckning' (sans) (rubrik)


%% \subsubsection{Rubriker}

%% Alla rubriker har stor bokstav i början, men i övrigt små bokstäver (utom i
%% namn, som ju alltid har stor bokstav i början oavsett om de kommer inuti en
%% rubrik eller ej). Rubriker avslutas inte med punkt.


%% \subsubsection{Underrubriker}


%% \subsection{Färg}

%% Hela dokumentet utom fram- och baksida har svart text på vit botten.

%% På fram- och baksida har texten Stockholms universitets blåa färg
%% (\texttt{\#002F5F}).


\subsection{Referenser}

Referenser skrivs enligt APA.

%% [eof]

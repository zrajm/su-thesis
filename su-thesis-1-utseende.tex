%% -*- mode: latex; ispell-dictionary: "svenska" -*-

\section{Stockholms universitets uppsatsmall}
\label{utseende}

Denna uppsatsmall bygger delvis på Stockholms universitets studentuppsatsmall
för Word, delvis på andra stilrekommendationer från Stockholms universitet som
jag kunnat hitta. Jag har också undersökt och jämfört ett tiotal
kandidatuppsatser skrivna vid Institutionen för lingvistik och därifrån hämtat
inspiration till sådant som jag inte hittat beskrivet.

\medskip

\begin{itemize}
\raggedright%
\item Stockholms universitets studentuppsatsmall (finns bara till Word):
  \\\href{%
    https://www.su.se/biblioteket/guider/guider/studentuppsatsmall%
  }{\texttt{%
      https://\linebreak[0]www.su.se/\linebreak[0]biblioteket/\linebreak[0]%
      guider/\linebreak[0]guider/\linebreak[0]studentuppsatsmall%
  }}.

\item Stockholms universitets grafiska manual:
  \\\href{%
    https://www.su.se/medarbetare/kommunikation/grafisk-manual%
  }{\texttt{%
      https://\linebreak[0]www.su.se/\linebreak[0]medarbetare/\linebreak[0]%
      kommunikation/\linebreak[0]grafisk-manual%
  }}.

\item Mallar och visuell identitet för avhandling:
  \\\href{%
    http://su.se/biblioteket/avhandlingsmallar%
  }{\texttt{%
      http://\linebreak[0]su.se/\linebreak[0]biblioteket/\linebreak[0]%
      avhandlingsmallar%
  }}.

\item Dokumentmall i Word för doktorsavhandlingar (PDF):
  \\\href{%
    https://www.su.se/polopoly\_fs/1.264229.1576661286!/menu/standard/%
    file/Instruktioner\%20SU\_Wordmall\_20180509.pdf%
  }{\texttt{%
      https://\linebreak[0]www.su.se/\linebreak[0]polopoly\_fs/\linebreak[0]1%
      \linebreak[0].264229\linebreak[0].1576661286\linebreak[0]!/\linebreak[0]%
      menu/\linebreak[0]standard/\linebreak[0]file/\linebreak[0]Instruktioner%
      \linebreak[0]\%20SU\linebreak[0]\_Wordmall\linebreak[0]\_20180509%
      \linebreak[0].pdf%
  }}.
\end{itemize}

%% \subsection{Typsnitt}

%% Brödtexten är satt i Times New Roman, medan titel, rubrik, underrubriker etc.
%% är satta i Verdana. På fram- och baksida är all text och universitetets logotyp
%% satta i Stockholms universitetets blåa färg, på övriga sidor är all text svart.

%% %% Brödtext (12 uppsatser):
%% %%  Blankrad: 11 / Indrag: 1 (Bark 2018)
%% %%  Vänsterjusterad: 7, Spärrad: 5 (Bäckström 2015, Lyxell 2016, Petersdotter
%% %%                                 2018, Bark 2018, Klintberg 2018)

%% \subsubsection{Times New Roman}

%% Brödtexten används för den huvudsakliga texten i uppsatsen. Den används också
%% för texten i styckena för sammanfattning/abstract, och nyckelord/keywords. Den
%% används däremot \emph{inte} för innehållsförteckning eller i textrutorna på
%% fram- och baksida.

%% %\textbf{Styckeindrag} -- Inget.


\subsection{Rubriker och brödtext}

Sidmarginalerna är 2,5~cm på alla fyra sidor. Rubriker är vänsterjusterade, och
satta i Verdana, brödtext har rak högermarginal och är satt i Times New Roman.
Stycken är separerade med blankrad, utan styckesindrag.

\medskip

\begin{tabular}{p{.34\textwidth} p{.66\textwidth}}
  \toprule
  {\sffamily\textbf{Typ}} &
  {\sffamily\textbf{Beskrivning}} \\
  \midrule
%%% Wordmall: 'Heading 1' (sans)
  {\sffamily\Large{}1~Rubrik} &
  Numrerad kapitelrubrik. Verdana 26~punkter. Vänsterjusterad.
  % I Libreoffice: (Line spacing "at least" 15~punkter.)
  Avstånd ovanför 36~punkter, avstånd nedanför 24~punkter.
  \\
  \midrule
%%% Wordmall: 'Heading 2' (sans) -- \subsection{}
  {\sffamily\bfseries\large{}1.1~Underrubrik} &
  Numrerad underrubrik. Verdana 16~punkter fetstil. Vänsterjusterad.
  % I Libreoffice: (Line spacing "at least" 15~punkter.)
  Avstånd ovanför 24~punkter, avstånd nedanför 6~punkter.
  \\
  \midrule
%%% Wordmall: 'Heading 3' (sans) -- \subsubsection{}
  {\sffamily\bfseries\normalsize{}1.1.1~Under-underrubrik} &
  Numrerad under-underrubrik. Verdana 11~punkter fetstil. Vänsterjusterad.
  % I Libreoffice: (Line spacing "at least" 15~punkter.)
  Avstånd ovanför 12~punkter, avstånd nedanför 6~punkter.
  \\
  \midrule
%%% Wordmall: 'Default Style'
  %% Set with:
  %%   \documentclass[11pt]{article} % set base font size
  %%   \setlength{\parindent}{0cm}   % remove 1st line of paragraph indentation
  %%   \setlength{\parskip}{6pt}     % add vertical space between paragraphs
  Brödtext &
  Times New Roman 11~punkter. Vänsterjusterad.
  % I Libreoffice: (Line spacing: Single.)
  Avstånd ovanför 0~punkter, avstånd nedanför 6~punkter.
  \\
  \bottomrule
%% %%% Wordmall: 'Header'
%%   Sidhuvud &
%%   \hl{FIXME}
%%   \\
%%   \midrule
%% %%% Wordmall: 'Footer'
%%   Sidfot &
%%   \hl{FIXME}
%%   \\
%%   \midrule
\end{tabular}

\subsection{Fram- och baksida}

Texten på fram- och baksidor är helt och hållet satt i Verdana (dvs
sans-seriff), och allt (text och logotyp) har Stockholms universitets blå färg.

\medskip

\begin{tabular}{p{.34\textwidth} p{.66\textwidth}}
  \toprule
  {\sffamily\textbf{Typ}} &
  {\sffamily\textbf{Beskrivning}} \\
  \midrule
%%% Wordmall: 'Uppsatstitel' (sans, blå)
  {\sffamily\raggedright\LARGE\color{sublue}Titel} &
  Uppsatstiteln på framsidan. Verdana 28~punkter rak. Vänsterjusterad.
  % I Libreoffice: (Line spacing "at least" 15~punkter.)
  \\
  \midrule
%%% Wordmall: 'Undertitel' (sans, blå)
  {\color{sublue}\sffamily\large{}Undertitel} &
  Undertitel och författarnamn på framsidan. Verdana 14~punkter rak.
  Vänsterjusterad.
  % I Lireoffice: (Line spacing "at least" 24~punkter.) Avstånd ovanför
  % 6~punkter, avstånd nedanför 28~punkter.
  \\
  \midrule
%%% Wordmall: 'Textruta' (sans, blå)
  {\color{sublue}\sffamily\small{}Textruta} &
  Textruta med tryckinformation längst ned till vänster på fram- och baksida.
  Verdana 8~punkter rak. Vänsterjusterad.
  % I Libreoffice: (Line spacing "fixed" 14~punkter.) Avstånd ovanför
  % 0~punkter, avstånd nedanför 0~punkter.
  \\
  \bottomrule
\end{tabular}


\subsection{Sammanfattningssida}

Detta är sidan efter framsidan. Den innehåller uppsats titel, undertitel och
författarnamn följt av en sammanfattningen och nyckelorden, och därefter den
engelska sammanfattningen (eller 'abstract') och nyckelorden.

Om språket är svenska (specificerat med \verb|\usepackage[swedish]{babel}|) så
visas den svenska titeln, och den svenska sammanfattningen före det engelska
abstractet, i annat fall visas den engelska titeln och det engelska abstractet
före den svenska sammanfattningen.

\medskip

\begin{tabular}{p{.34\textwidth} p{.66\textwidth}}
  \toprule
  {\sffamily\textbf{Typ}} &
  {\sffamily\textbf{Beskrivning}} \\
  \midrule
%%% Wordmall: 'Uppsatstitel sidan2' (sans) -- \section*{}
  {\sffamily\Large{}Titel} &
  %% Verdana 26~punkter. Vänsterjusterad. (Line spacing "at least"
  %% 15~punkter.) Avstånd ovanför 36~punkter, avstånd nedanför 24~punkter \\
  Uppsatstiteln (samma som '1 Rubrik').
  \\
  \midrule
%%% Wordmall: 'Undertitel sidan2' (sans)
  {\sffamily\textbf{Undertitel}} &
  Används på sammanfattningssidan till undertitel och författarnamn på.
  Verdana 11~punkter, fetstil. Vänsterjusterad.
  % I Libreoffice: (Line spacing "at least" 15~punkter.)
  Avstånd ovanför 36~punkter, avstånd nedanför 18~punkter (samma som
  under-underrubrik men med större avstånd ovanför och under).
  \\
  \midrule
%%% Wordmall: 'Sammanfattning' (sans) -- \section*{}
  {\sffamily\Large{}Rubrik} &
  Onumrerad rubrik till sammanfattning/abstract
  %% Verdana 26~punkter. Vänsterjusterad. (Line spacing "at least"
  %% 15~punkter.) Avstånd ovanför 36~punkter, avstånd nedanför 24~punkter
  (samma som '1 Rubrik').
  \\
  \midrule
%%% Wordmall: 'Nyckelord' (sans) samma som 'Heading 3'?)
  {\sffamily\textbf{Nyckelordsrubrik}} &
  Onumrerad under-underrubrik. Verdana 11~punkter fetstil. Vänsterjusterad.
  % I Libreoffice (Line spacing "at least" 15~punkter.)
  Avstånd ovanför 24~punkter, avstånd nedanför 6~punkter (samma som
  under-underrubrik, men med dubbelt så stort avstånd ovanför).
  \\
  \bottomrule
\end{tabular}


%%% Wordmall: 'Uppsatstitel' (sans, blå)
%% \textbf{Titel} -- Används bara på framsidan, till uppsatstiteln. Verdana
%% 18~punkter. Vänsterjusterad. (Line spacing "at least" 15~punkter.) Färg: blå.

%% %%% Wordmall: 'Undertitel' (sans, blå)
%% \textbf{Undertitel} -- Används bara på framsidan till undertitel och
%% författarnamn. Verdana 14~punkter, rak. Vänsterjusterad. (Line spacing "at
%% least" 24~punkter.) Avstånd ovanför 6~punkter, avstånd nedanför 28~punkter.
%% Färg: blå.

%% %%% Wordmall: 'Textruta' (sans, blå)
%% \textbf{Textruta} -- Används bara på fram- och baksida till tryckinformation
%% (textrutan längst ned till vänster). Verdana 8~punkter, rak.
%% Vänsterjusterad. (Line spacing "fixed" 14~punkter.) Avstånd ovanför 0~punkter,
%% avstånd nedanför 0~punkter. Färg: blå.

%% %%% Wordmall: 'Uppsatstitel sidan2' (sans)
%% \textbf{Titel sida 2} -- Används bara på sammanfattningssidan. Verdana
%% 26~punkter. Vänsterjusterad. (Line spacing "at least" 15~punkter.) Avstånd
%% ovanför 36~punkter, avstånd nedanför 24~punkter.

%% %%% Wordmall: 'Undertitel sidan2' (sans)
%% \textbf{Undertitel sida 2} -- Används bara på sammanfattningssidan till
%% undertitel och författarnamn på framsidan. Verdana 11~punkter, fetstil.
%% Vänsterjusterad. (Line spacing "at least" 15~punkter.) Avstånd ovanför
%% 36~punkter, avstånd nedanför 18~punkter.

%% %%% Wordmall: 'Heading 1' (sans)
%% \textbf{Rubrik (rubrik 1)} -- Numrerad kapitelrubrik.

%% %%% Wordmall: 'Heading 2' (sans)
%% \textbf{Underrubrik (rubrik 2)} -- Numrerad underrubrik.

%% %%% Wordmall: 'Heading 3' (sans)
%% \textbf{Under-underrubrik (rubrik 2)} -- Numrerad under-underrubrik.

%% %%% Wordmall: 'Sammanfattning' (sans)
%% \textbf{Sammanfattning}

%% %%% Wordmall: 'Nyckelord' (sans) samma som 'Heading 3'?)
%% \textbf{Nyckelordsrubrik}

%% %%% Wordmall: 'Innehållsförteckning' (sans) (rubrik)


%% \subsubsection{Rubriker}

%% Alla rubriker har stor bokstav i början, men i övrigt små bokstäver (utom i
%% namn, som ju alltid har stor bokstav i början oavsett om de kommer inuti en
%% rubrik eller ej). Rubriker avslutas inte med punkt.


%% \subsubsection{Underrubriker}


%% \subsection{Färg}

%% Hela dokumentet utom fram- och baksida har svart text på vit botten.

%% På fram- och baksida har texten Stockholms universitets blåa färg
%% (\texttt{\#002F5F}).


\subsection{Referenser}

Referenserna formateras automatiskt enligt APA-standard, baserat på den
BibTeX-information du angivit i din \verb|.bib|-fil.

Placera följande kommandon på den plats i din \verb|.tex|-fil där du vill ha
dina referenser. En onumrerad rubrik på ditt valda språk kommer automatiskt
infogas före referenserna (på svenska 'Referenser').

\begin{verbatim}
\nocite{*}                               % visa även oanvända referenser
\addcontentsline{toc}{section}{\refname} % inkludera i innehållsförteckning
\bibliography{dina_referenser.bib}       % ladda 'dina_referenser.bib'
\end{verbatim}

Om du använder \verb|\nocite{*}| så kommer referenskapitlet inkludera även de
referenser som du inte refererat till i din text, men dessa visas i grått för
att göra det lättare att se vilka det är.

\verb|\addcontentsline|-kommandot lägger till referenskapitlet i din uppsats
innehållsförteckning. Notera att detta kommando måste komma precis före
\verb|\bibliography|-kommandot för att sidreferensen i innehållsförteckningen
ska peka på rätt sida.

%% [eof]

%% -*- mode: latex; ispell-dictionary: "svenska" -*-

\section{Att använda \LaTeX}
\label{latex}

Detta är en uppsatsmall för \LaTeX\ för Stockholms universitet. Den kan
användas med \XeLaTeX\ på din egen dator eller med Overleaf
(\texttt{\href{https://overleaf.com/edu/su}{https://\linebreak[0]{}overleaf\linebreak[0]{}.com/\linebreak[0]{}edu/\linebreak[0]{}su}})
-- som student på Stockholms universitet har du automatiskt tillgång Overleaf.

Denna mall måste kompileras med \XeLaTeX, vilket du behöver slå på manuellt om
du använder Overleaf.

\subsection{Overleaf}


\subsection{På egen dator}

\subsubsection{Installation av program}

Denna uppsatsmall kräver fungerar inte med PDF\LaTeX, utan kräver XeLaTeX eller
\LuaLaTeX\footnote{Stilmallen är inte testad med \LuaLaTeX{} men det borde
  fungera.} för att fungera. Detta beror på att mallen använder sig av
\LaTeX-paketet \texttt{fontspec} för att ladda de Truetype-typsnitt som
specificeras av Stockholms stilrekommendationer och detta paket fungerar inte
under PDF\LaTeX.

Jag rekommenderar att du använder bekvämlighetsverktyget
\texttt{latexmk}\footnote{Dokumentationen till kommandot \texttt{latexmk} finns
  här: \url{https://mg.readthedocs.io/latexmk.html}. } för att kompilera din
\LaTeX-kod. Om du gör det kan du använda den \texttt{latexmkrc}-fil som följer
med uppsatsmallen för att om det som behövs för kompilering.

Innan du börjar behöver du se till att alla relevanta program är installerade
på datorn. Om du kör Linux (Ubuntu, Debian eller likanande) så kan du kanvända
följande kommando för att installera:

\begin{verbatim}
sudo apt install texlive-xetex latexmk
\end{verbatim}



\subsubsection{Kompilering}

Det enklaste sättet att kompilera din uppsats är:

\begin{verbatim}
latexmk su-thesis.tex
\end{verbatim}

De inställningar som behövs till \texttt{latexmk} följer med uppsatsmallen och
finns i en fil som heter \texttt{latexmkrc}. Om du installerar denna
uppsatsmall i en underkatalog så vill du förmodligen också kopiera
\texttt{latexmkrc} till katalogen där du har din uppsats.

Om du inte använder \texttt{latexmk} utan insisterar på att kompilera
själv så motsvarar detta följande kommandon (Ja! -- \texttt{xelatex} behöver
vanligtvis köras tre gånger för att vara säker på att både innehållsförteckning
och referenser stämmer överens med texten).

\begin{verbatim}
export TEXINPUTS=".//:$TEXINPUTS"
xelatex su-thesis.tex
bibtex su-thesis.bib
xelatex su-thesis.tex
xelatex su-thesis.tex
\end{verbatim}

\noindent Då och då råkar man skriva något galet i \LaTeX-koden, och kompileringen
stannar mitt i. Då får man upp en prompt, med ett frågetecken i början på en ny
rad. -- När detta händer skriv 'x' (och tryck enter) för att avsluta. Rätta
felet i din \texttt{.tex}-fil och kompilera om.


\subsection{Att skriva referenser}

För att skriva en referens i din text använder antingen kommandot
\texttt{\textbackslash{}citep\{\}} (för en parentetisk referens) eller
kommandot \texttt{\textbackslash{}citet\{\}} (för en referens utan parentes
runt författarnamnet).

\begin{table}[ht]
  \centering
  \renewcommand{\arraystretch}{1.25}%
  \begin{tabular}[t]{lcc}
    \toprule
    {\sffamily\textbf{Kommando}} &
    {\sffamily\textbf{Beskrivning}} &
    {\sffamily\textbf{Exempel}} \\
    \midrule
    \texttt{\textbackslash{}citep\{källa\}} &
    med parentes &
    \citep{bergman+wallin-2001} \\
    \texttt{\textbackslash{}citet\{källa\}} &
    utan parentes &
    \citet{bergman+wallin-2001} \\
    \midrule
    \texttt{\textbackslash{}citep\{källa1,källa2\}} &
    flera källor &
    \citep{bergman+wallin-2001,bergman-1977} \\
    \texttt{\textbackslash{}citet\{källa1,källa2\}} &
    flera källor &
    \citet{bergman+wallin-2001,bergman-1977} \\
    \midrule
    \texttt{\textbackslash{}citep[1--2]\{källa1\}} &
    sidnummer &
    \citep[1--2]{bergman-1977} \\
    \texttt{\textbackslash{}citep[jmf][]\{källa1\}} &
    text före &
    \citep[jmf][]{bergman-1977} \\
    \texttt{\textbackslash{}citep[jmf][1--2]\{källa1\}} &
    text + sidref. &
    \citep[jmf][1--2]{bergman-1977} \\
    \midrule
    \texttt{\textbackslash{}citealp\{källa\}} &
    som \texttt{\textbackslash{}citep\{källa\}} &
    \citealp{bergman+wallin-2001} \\
    \texttt{\textbackslash{}citealt\{källa\}} &
    som \texttt{\textbackslash{}citet\{källa\}} &
    \citealt{bergman+wallin-2001} \\
    \bottomrule
  \end{tabular}
  \caption{Några \LaTeX-kommandon för att skriva referenser (från
    \texttt{natbib}-paketet)}.
\end{table}

\texttt{\textbackslash{}citealp} och \texttt{\textbackslash{}citealt}
kommandona fungerar som \texttt{\textbackslash{}citep} och
\texttt{\textbackslash{}citet} men utelämnar de omgivande parenteserna. Ibland
kan detta vara nödvändigt, som tex om man vill ange två källor med sidreferens
inom parentes. I detta fall kan man skriva
"\texttt{(\textbackslash{}citealp[55]\{källa1\} och
  \textbackslash{}citealp[25]\{källa2\})}".

Det finns ytterligare kommandon som kan användas för att skapa referenser, hur
de används finns beskrivet i
\underline{\href{http://ftp.acc.umu.se/mirror/CTAN/macros/latex/contrib/natbib/natbib.pdf}{\texttt{natbib}-paketets referensmanual}}.

%% [eof]

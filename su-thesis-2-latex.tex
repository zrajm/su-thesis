%% -*- mode: latex; ispell-dictionary: "svenska" -*-

\section{Att använda {\rmfamily\LaTeX}}
\label{latex}

Detta är en uppsatsmall för \LaTeX\ för Stockholms universitet. Den kan
användas med \XeLaTeX\ på din egen dator eller med Overleaf
(\texttt{\href{https://overleaf.com/edu/su}{https://\linebreak[0]{}overleaf\linebreak[0]{}.com/\linebreak[0]{}edu/\linebreak[0]{}su}})
-- som student på Stockholms universitet har du automatiskt tillgång Overleaf.

Denna mall måste kompileras med \XeLaTeX, vilket du slå på manuellt i Overleaf.

\subsection{Overleaf}


\subsection{På egen dator}

\subsubsection{Installation av program}

Denna uppsatsmall kräver kommandot \texttt{xelatex} (det traditionella
kommandot \texttt{pdflatex} fungerar inte. \hl{FIXME: fungerar inte
pdflatex?}). Denna guide utgår också ifrån bekvämlighetsverktyget
\texttt{latexmk}.

\noindent Om du kör Linux (med Debian eller Ubuntu) kör du följande kommandon (som
användaren 'root') för att installera:

\begin{verbatim}
apt install texlive-xetex
apt install latexmk
\end{verbatim}


\subsubsection{Kompilering}

Det enklaste sättet att kompilera din uppsats är:

\begin{verbatim}
latexmk -xelatex su-thesis.tex
\end{verbatim}

\noindent Om du inte använder \texttt{latexmk} utan insisterar på att kompilera
själv så motsvarar detta följande kommandon (Ja! -- \texttt{xelatex} behöver
vanligtvis köras tre gånger för att vara säker på att både innehållsförteckning
och referenser stämmer överens med texten).

\begin{verbatim}
xelatex su-thesis.tex
bibtex su-thesis.bib
xelatex su-thesis.tex
xelatex su-thesis.tex
\end{verbatim}

\noindent Då och då råkar man skriva något galet i \LaTeX-koden, och kompileringen
stannar mitt i. Då får man upp en prompt, med ett frågetecken i början på en ny
rad. -- När detta händer skriv 'x' (och tryck enter) för att avsluta. Rätta
felet i din \texttt{.tex}-fil och kompilera om.


%% [eof]

%% -*- mode: latex; ispell-dictionary: "svenska" -*-

\section{Vilken \LaTeX{} ska jag använda?}
\label{latex}

Detta är en uppsatsmall för \LaTeX{} för Stockholms universitet. Den kan
användas med \XeLaTeX{} på din egen dator eller med webbverktyget Overleaf
(\href{https://overleaf.com/edu/su}{\url{overleaf.com/edu/su}}) -- som student
på Stockholms universitet har du automatiskt tillgång Overleaf.

Denna uppsatsmall kräver \XeLaTeX{} eller \LuaLaTeX{}\footnote{Uppsatsmallen är
  inte testad med \LuaLaTeX{} men det borde fungera.} för att fungera. Den
fungerar inte med PDF\LaTeX{}. Detta beror på att bara \XeLaTeX{} och \LuaLaTeX
stödjer Truetype-typsnitt, vilket behövs för att ladda de typsnitt som
Stockholms universitets stilmanual rekommenderar.


\subsection{Overleaf}

Om du använder Overleaf måste du ändra inställningarna så att ditt dokument
kompileras med \XeLaTeX{} (om du använder det förvalda alternativet PDF\LaTeX{}
så kommer ditt dokument inte kunna kompilera).


\subsection{På egen dator}

\subsubsection{Installation av program}

Jag rekommenderar att du använder bekvämlighetsverktyget
\texttt{latexmk}\footnote{Dokumentationen till kommandot \texttt{latexmk} finns
  här: \url{https://mg.readthedocs.io/latexmk.html}. } för att kompilera din
\LaTeX{}-kod. Om du gör det kan du använda den \texttt{latexmkrc}-fil som
följer med uppsatsmallen för att om det som behövs för kompilering.

Innan du börjar behöver du se till att alla relevanta program är installerade
på datorn. Om du kör Linux (Ubuntu, Debian eller likanande) så kan du kanvända
följande kommando för att installera:

\begin{verbatim}
sudo apt install texlive-xetex latexmk
\end{verbatim}


\subsubsection{Kompilering}

Det enklaste sättet att kompilera din uppsats är:

\begin{verbatim}
latexmk su-thesis.tex
\end{verbatim}

De inställningar som behövs till \texttt{latexmk} följer med uppsatsmallen och
finns i en fil som heter \texttt{latexmkrc}. Om du installerar denna
uppsatsmall i en underkatalog så vill du förmodligen också kopiera
\texttt{latexmkrc} till katalogen där du har din uppsats.

Om du inte använder \texttt{latexmk} utan insisterar på att kompilera
själv så motsvarar detta följande kommandon (Ja! -- \texttt{xelatex} behöver
vanligtvis köras tre gånger för att vara säker på att både innehållsförteckning
och referenser stämmer överens med texten).

\begin{verbatim}
xelatex su-thesis.tex
bibtex su-thesis.bib
xelatex su-thesis.tex
xelatex su-thesis.tex
\end{verbatim}

\noindent Då och då råkar man skriva något galet i \LaTeX{}-koden, och
kompileringen stannar mitt i. Då får man upp en prompt, med ett frågetecken i
början på en ny rad. -- När detta händer skriv 'x' (och tryck enter) för att
avsluta. Rätta felet i din \texttt{.tex}-fil och kompilera om.

%% [eof]

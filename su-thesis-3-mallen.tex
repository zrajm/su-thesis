%% -*- mode: latex; ispell-dictionary: "svenska" -*-

\section{Att använda mallen}
\label{mallen}


\subsection{Ladda mallen}
\label{variabler}

För att ladda mallen använd:

\begin{verbatim}
\usepackage{su-thesis}
\end{verbatim}


\subsection{Välja språk}
\label{sprak}

\noindent Språket på uppsatsen ställer man in med hjälp av paketet
\texttt{babel}. När du byter språk så kommer byts fram- och baksida ut mot
information på det relevanta språket (men resten av uppsatsen får du förstås
översätta själv). :)

\begin{verbatim}
\usepackage[swedish]{babel} % Swedish
%% \usepackage[UKenglish]{babel} % British English
%% \usepackage[USenglish]{babel} % American English
\end{verbatim}

\noindent\textbf{Notera:} När du bytt språk behöver du radera alla tempfiler
och kompilera om. Om du använder \texttt{latexmk} är detta enkelt.

\begin{verbatim}
latexmk -xelatex su-thesis.tex -C  % radera alla tempfiler
latexmk -xelatex su-thesis.tex     % kompilera om
\end{verbatim}


\subsection{Mallens variabler}
\label{variabler}

Mallen har ett antal variabler som man sätter med \verb|\suset[SPRÅK]{NAMN}|
(för tillfället kan bara \texttt{swedish} och \texttt{english} användas som
språk).

\textbf{Notera:} Du behöver ladda mallen innan du kan sätta variablerna.

\begin{verbatim}
%% Always used
\suset{author}{Förnamn Efternamn}
\suset{supervisor}{Förnamn1 Efternamn1[, Förnamn2 Efternamn2]}
\suset[swedish]{title}{Titel på svenska}
\suset[english]{title}{Title in English}
\suset[swedish]{abstract}{Sammanfattning på svenska}
\suset[english]{abstract}{Abstract in English}

%% Swedish metadata (only used if language is Swedish)
\suset[swedish]{subtitle}{Svensk undertitel}
\suset[swedish]{keywords}{nyckelord1, nyckelord2, nyckelord3, \ldots}
\suset[swedish]{department}{Institutionen för lingvistik}
\suset[swedish]{thesistype}{Examensarbete 15 hp}
\suset[swedish]{course}{Lingvistik -- kandidatkurs, LIN600}
\suset[swedish]{program}{Kandidatprogrammet i lingvistik 180 hp}
\suset[swedish]{semester}{Vårterminen 2020}

%% English metadata (only used if language is English)
\suset[english]{subtitle}{English Subtitle}
\suset[english]{keywords}{keyword1, keyword2, keyword3, \ldots}
\suset[english]{department}{Department of Linguistics}
\suset[english]{thesistype}{Bachelor's Thesis 15 ECTS credits}
\suset[english]{course}{Linguistics -- Bachelor's Course, LIN600}
\suset[english]{program}{Bachelor's Programme in Linguistics 180 ECTS credits}
\suset[english]{semester}{Spring semester 2020}
\end{verbatim}


\subsection{Fonetiska tecken (IPA)}
\label{ipa}

\hl{FIXME: Lägg till en \texttt{\textbackslash{}suipa\{...\}} funktion.}

Det finns en IPA-skrivmaskin på nätet som man kan använda för att skriva
IPA-symboler:
\texttt{\href{https://ipa.typeit.org/full/}{https://\linebreak[0]{}ipa\linebreak[0]{}.typeit\linebreak[0]{}.org/\linebreak[0]{}full/}}


\subsection{Ta bort inledande/avslutande sidor}
\label{genererade}

Mallen använder interna kommandon för att genera inledande och avslutande
sidor. För att ta bort en eller flera av dessa kan man definiera om de
relevanta kommandona så att det inte generar någon effekt. Detta görs med hjälp
av \LaTeX-kommandot \verb|\renewcommand|. Följande kommandon kan användas:

\begin{verbatim}
\renewcommand{\sumakefrontpage}{}     % ta bort framsida
\renewcommand{\sumakeabstractpage}{}  % ta bort sammanfattningssida
\renewcommand{\sutableofcontents}{}   % ta bort innehållsförteckning
\renewcommand{\sumakebackpage}{}      % ta bort baksida
\end{verbatim}

\noindent\textbf{Notera:} Raden med \verb|\renewcommand| måste komma efter
raden som laddar mallen.

%% [eof]

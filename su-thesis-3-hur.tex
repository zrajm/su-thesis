%% -*- mode: latex; ispell-dictionary: "svenska" -*-

\section{Hur använda \LaTeX{}?}
\label{hur}

Uppsatsmallen \texttt{su-thesis} är tänkt att vara en hjälp i uppsatsskrivandet
så långt som möjligt, men det finns också andra bra vanor som inte kan byggas
in i en modul. Jag försöker här summera några av de viktigaste lärdomarna jag
gjort själv.

\LaTeX{} är populärt och har funnits lääänge (sedan 1984) vilket innebär att
det finns en uppsjö tillägg och extrapaket för allt mellan himmel och jord. Så
om det är någonting du saknar, eller funderar över -- googla!

Websajten \href{https://ctan.org/}{\url{ctan.org}} (\emph{Comprehensive Tex
  Archive Network}) har massor med dokumentation, så lägg till "ctan" i din
googlesökning om du letar efter något du gissar skulle kunna stå i
dokumentationen.


\subsection{En grundplåt}
\label{grundplåt}

Ett minimalt startdokument för din uppsats kan se ut såhär:

\begin{verbatim}
\documentclass[11pt]{article}
\usepackage[swedish]{babel}
\usepackage{su-thesis}

\suset{author}{Shevek}
\suset{supervisor}{Sabul}
\suset[swedish]{title}{Principer för samtidighet}
\suset[swedish]{subtitle}{Om tiden och samtidighetens natur}
\suset[english]{title}{Principles of Simultaneity}
\suset[english]{subtitle}{On the Nature of Time and Simultaneity}
\suset[swedish]{abstract}{...}
\suset[english]{abstract}{...}

\begin{document}
\include{kapitel-1-sekvens}      % laddar 'kapitel-1-sekvens.tex'
\include{kapitel-2-samtidighet}
\include{kapitel-3-referenser}
\end{document}
\end{verbatim}

Ovanstående dokument använder \texcommand{include} för att inkludera de olika
kapitlen. Varje deldokument som laddas på detta sätt börjar automatiskt på en
ny sida så det passar bra för just kapitel. \LaTeX{} kompilerar också de olika
deldokumenten varförsig vilket gör att det går fortare att kompilera om
helheten om man bara ändrat i en eller ett par av filerna.

Ett annat sätt man kan snabba upp kompilerandet på är genom att bara kompilera
det kapitel man för tillfället jobbar med. Om du tex arbetar med det första
kapitlet i ovanstående dokument, så kan du lägga till följande kommando
(ovanför \texcommand[document]{begin}) för att bara kompilera det första
kapitlet och kapitlet med referenser.

\begin{verbatim}
\includeonly{
  kapitel-1-sekvens,
  kapitel-3-referenser,
}
\end{verbatim}

När du sedan är klar och vill kompilera alltihop, så tar du bara bort
\texcommand{includeonly} och kompilerar om.

För mer information att välja språk se \autoref{språk}, för mer information om
vilka \texcommand{suset} variabler mallen har se \autoref{variabler}.


\subsection{Att referera inom texten}
\label{länkar}

Ibland vill man referera till en annan plats i sin egen text. Detta gör man
genom att placera ut en etikett \texcommand{label} på stället man vill referera
till. Därefter kan man skapa en länkar med kommandot \texcommand{autoref} med
samma etikett, länktexten anpassar sig automatiskt beroende på om det är en
länk till figur/tabell eller en kapitelrubrik (i det här dokumentet blir tex
\texcommand[tab-ipa]{autoref} → "\autoref{tab-ipa}" och
\texcommand[citera]{autoref} → "\autoref{citera}").

\textbf{Rekommendation:} Placera alltid ut \texcommand{label} direkt efter alla
rubriker, underrubriker och \texcommand{caption} i tabeller och figurer. Tänk
på att hålla etiketten kort men ändå unik inom ditt dokument. Det är vanligt
att låta tabeller och figurers etikett börja på "\texttt{tab-}" respektive
"\texttt{fig-}" eller liknande.


\subsection{Bilder och figurer}
\label{bilder}

Om du vill lägga till en bild i din uppsats, skapa en katalog som heter
"\texttt{images/}" och lägg bilden där. Därefter använder du \LaTeX{}-kommandot
\texcommand[FILNAMN]{includegraphics} i din uppsats för att inkludera bilden.
(Notera att filnamnet inte ska innehålla namnet på bildkatalogen
"\texttt{images/}".) En bild utgör oftast en figur i uppsatsen. En enkel figur
med en bild kan se ut såhär:

\begin{multicols}{2}
\null \vfill
\noindent\begin{verbatim}
\begin{figure}[ht]
  \centering
  \includegraphics[
    width=.666\linewidth
  ]{su-logo-sv.png}

  \caption{Bildtext till figuren.}
  \label{fig-test}
\end{figure}
\end{verbatim}

\vfill \null
\columnbreak

{
  \centering
  \hypertarget{fig-test}{%
    \includegraphics[width=.666\linewidth]{su-logo-sv.png}%
  } \\
  \vspace{.5em}
  Figur 1: Bildtext till figuren. \\ % '\\' needed for \centering to work
}
\end{multicols}

En figur börjar med \texcommand[figure]{begin}, en tabell med
\texcommand[table]{begin} och slutar med \texcommand{end}. Du behöver också
ange en bildtext med \texcommand{caption} och en etikett med
\texcommand{label}. Etiketten är ingenting som läsaren ser, utan används när du
vill referera till den i texten med kommandot \texcommand{autoref} (se
\autoref{länkar}). Notera att etiketten alltid måste komma efter bildtexten.

Figurer och tabeller är "flytande" vilket innebär att \LaTeX{} kan flytta runt
dem och placera dem där det ser bäst ut. Om en tabell eller figur inte får
plats på sidan flyttas den oftast till nästa sida. Kom därför ihåg att hänvisa
till din figur/tabell i texten. Figuren ovan har etiketten "\texttt{fig-test}"
så om vi vill referera till den använder vi kommandot
\texcommand[fig-test]{autoref} vilket resulterar i en länk med texten
"\hyperlink{fig-test}{Figur 1}".

Ibland kan en tabell eller figur växa i storlek och det kan vara praktiskt att
bryta ut den och ha den i en egen fil för att göra det lättare att navigera i
källkoden. I dessa fall brukar jag skapa en katalog "\texttt{tables/}"
och/eller "\texttt{figures/}" och lägga mina tabeller/figurer där och sedan
använda \LaTeX{}-kommandot \texcommand{input} för att läsa in tabellen där den
ska vara. (Jag brukar behålla bildtext och \texcommand{label} i
mammadokumentet, och bara använda \texcommand{input} för själva
tabellen/figuren i fråga.) Resultatet brukar se ut ungefär såhär:

\begin{verbatim}
\begin{table}
  \caption{Svenska konsonanter.}
  \label{tab-konsonanter}
  \input{tables/konsonanter}       % laddar 'tables/konsonanter.tex'
\end{table}
\end{verbatim}

\textbf{Notera:} Här använder vi kommandot \texcommand{input} till skillnad
från \texcommand{include} (som vi använde i \autoref{grundplåt}).
\texcommand{input} lägger inte till någon sidbrytning före/efter och det
kompileras om varje gång (även om ingen förändring skett) så det är bättre
lämpat för mindre deldokument (som tex tabeller/figurer) medan
\texcommand{include} passar bättre för hela kapitel.


\subsection{Att citera en referens}
\label{citera}

För att skriva en referens i din text använder antingen kommandot
\texttt{\textbackslash{}citep\{\}} (för en parentetisk referens) eller
kommandot \texttt{\textbackslash{}citet\{\}} (för en referens utan parentes
runt författarnamnet).

\begin{table}[ht]
  \centering
  \renewcommand{\arraystretch}{1.25}%
  \begin{tabular}[t]{lcc}
    \toprule
    {\sffamily\textbf{Kommando}} &
    {\sffamily\textbf{Beskrivning}} &
    {\sffamily\textbf{Exempel}} \\
    \midrule
    \texttt{\textbackslash{}citep\{källa\}} &
    med parentes &
    \citep{bergman+wallin-2001} \\
    \texttt{\textbackslash{}citet\{källa\}} &
    utan parentes &
    \citet{bergman+wallin-2001} \\
    \midrule
    \texttt{\textbackslash{}citep\{källa1,källa2\}} &
    flera källor &
    \citep{bergman+wallin-2001,bergman-1977} \\
    \texttt{\textbackslash{}citet\{källa1,källa2\}} &
    flera källor &
    \citet{bergman+wallin-2001,bergman-1977} \\
    \midrule
    \texttt{\textbackslash{}citep[1--2]\{källa1\}} &
    sidnummer &
    \citep[1--2]{bergman-1977} \\
    \texttt{\textbackslash{}citep[jmf][]\{källa1\}} &
    text före &
    \citep[jmf][]{bergman-1977} \\
    \texttt{\textbackslash{}citep[jmf][1--2]\{källa1\}} &
    text + sidref. &
    \citep[jmf][1--2]{bergman-1977} \\
    \midrule
    \texttt{\textbackslash{}citealp\{källa\}} &
    som \texttt{\textbackslash{}citep\{källa\}} &
    \citealp{bergman+wallin-2001} \\
    \texttt{\textbackslash{}citealt\{källa\}} &
    som \texttt{\textbackslash{}citet\{källa\}} &
    \citealt{bergman+wallin-2001} \\
    \bottomrule
  \end{tabular}
  \caption{Några \LaTeX{}-kommandon för att skriva referenser (från
    \texttt{natbib}-paketet)}.
  \label{tab-cite-kommandon}
\end{table}

\texttt{\textbackslash{}citealp} och \texttt{\textbackslash{}citealt}
kommandona fungerar som \texttt{\textbackslash{}citep} och
\texttt{\textbackslash{}citet} men utelämnar de omgivande parenteserna. Ibland
kan detta vara nödvändigt, som tex om man vill ange två källor med sidreferens
inom parentes. I detta fall kan man skriva
"\texttt{(\textbackslash{}citealp[55]\{källa1\} och
  \textbackslash{}citealp[25]\{källa2\})}".

Det finns ytterligare kommandon som kan användas för att skapa referenser, hur
de används finns beskrivet i
\underline{\href{http://ftp.acc.umu.se/mirror/CTAN/macros/latex/contrib/natbib/natbib.pdf}{\texttt{natbib}-paketets referensmanual}}.

%% [eof]

%%
%% This table uses 'tabularx', 'array' and 'makecell' packages.
%%
%% * 'Tables in LATEX2ε: Packages and Methods' (docs on tables etc.)
%%   https://www.tug.org/pracjourn/2007-1/mori/mori.pdf
%%
%% * 'The tabularx package∗'
%%   http://mirrors.ibiblio.org/CTAN/macros/latex/required/tools/tabularx.pdf
%%
%%
%% Random Observations
%% ===================
%%
%%       \setlength{\extrarowheight}{40pt}%   % array:
%%       \setlength{\tabcolsep}{<length>}     % tabular: ½ of width between cols
%%       \setlength{\arraycolsep}{<length>}   % array:   ½ of width between cols
%%       %%\extracolsep{\fill}
%%
%%   Vertical padding is possible in a global way using @Herbert's answer. That
%%   is, to redefine the array stretch factor <factor> using
%%
%%       \renewcommand\arraystretch{.01}%      % (array)
%%
%%   \baselineskip -- Specifies minimum space between baselines successive
%%   lines in a paragraph. Is reset, for example, by font changes. The value in
%%   effect at end of a paragraph, is used for whole paragraph.
%%
%%       \setlength{\baselineskip}{0pt}%      % minimum line height (reset by font commands)
%%
%%   \baselinestretch is used to multiply \baselineskip. (Default: 1.0). Use
%%   this to change line height in a document, since its not reset by other
%%   commands.
%%
%%       \renewcommand{\baselinestretch}{0}%   % multiplier of line height (default: 1)
%%       \arraybackslash%                      % '\\' inside table
%%
%%
%% Column Specifications
%% =====================
%% These column specifications are used by 'tabularx'. Some of them are
%% imported from the 'array' package, and some from the (non-x) 'tabular'.
%%
%%     @{text}     Replace padding between this & previous column with <text>
%%     >{code}     (array) insert <code> at beginning of cell
%%     l           left aligned
%%     r           right aligned
%%     c           horizontally centered
%%     p{width}    justified
%%     m{width}    (array) vertically centered
%%     b{width}    (array) bottom align
%%     <{code}     (array) insert <code> at end of cell
%%     @{text}     Replace padding between this & next column with <text>
%%
%% Relative Column Widths ('tabularx')
%% ===================================
%% If there are two columns, then then all <\hsize>s added together should be
%% 2, but if you want differing widths you could use something like this (which
%% will cause the first column to be three times wider than the second):
%%
%%    \begin{tabularx}{\linewidth}{|%
%%      >{\hsize=1.5\hsize\linewidth=\hsize}c|%
%%      >{\hsize=0.5\hsize\linewidth=\hsize}c|}
%%
\centering%
{
  \large%
  \newcommand{\na}{\makecell{}}               % empty cell content
  \newcommand{\td}[1]{\makecell{#1}}
  \renewcommand{\theadfont}{\normalsize}      % small font in header
  \renewcommand\theadalign{cc}
  \renewcommand\cellalign{cc}
  \renewcommand\theadset{                     % (makecell)
    \renewcommand\arraystretch{.6}%           % (array) line height
    %\renewcommand\theadalign{c}
  }%
  \renewcommand{\tabularxcolumn}[1]{>{}m{#1}}
  \newcolumntype{Z}{%                       % phoneme column
    >{\centering\arraybackslash}X%
  }%
  \newcolumntype{P}{%                       % Placeholder column
    @{}%                                    %   suppress left margin
    c%                                      %   horizontally centered text
    @{}%                                    %   suppress right margin
  }%
  %% Left side table headers.
  %%
  %% Leading & trailing space should be of same width here. But something is
  %% adding approx .2em space on the left side, so that's from the left hand
  %% side subtracted below. (Specifying an empty @{} will suppress the
  %% default space between columns.)
  %%
  \newcolumntype{H}{%             % Header
    @{\hspace{.3em}}%             %   left margin
    c%                            %   horizontally centered text
    @{\hspace{.5em}}%             %   right margin
  }%
  \begin{tabularx}{\linewidth}{P|H|ZZ|ZZ|ZZ|ZZ|ZZ|ZZ|ZZ|}
    \hline%---------------------------------------------------------------------
    \makecell{\rule{0pt}{1.5em}} &
                                                & % header = 3 columns wide
    \multicolumn{2}{c|}{\thead{Bilabial}}       &
    \multicolumn{2}{c|}{\thead{Labio-\\dental}} &
    \multicolumn{2}{c|}{\thead{Dental}}         &
    \multicolumn{2}{c|}{\thead{Alveolar}}       &
    \multicolumn{2}{c|}{\thead{Palatal}}        &
    \multicolumn{2}{c|}{\thead{Velar}}          &
    \multicolumn{2}{c|}{\thead{Glottal}} \\
    \hline%---------------------------------------------------------------------
    \makecell{\rule{0pt}{1.5em}} &
    \thead{Klusil} &
    \td{p} & \td{b} &  % bilabial
    \na    & \na    &  % labiodental
    \td{t} & \td{d} &  % dental
    \na    & \na    &  % alveolar
    \na    & \na    &  % palatal
    \td{k} & \td{ɡ} &  % velar
    \na    & \na    \\ % glottal
    \hline%---------------------------------------------------------------------
    \makecell{\rule{0pt}{1.5em}} &
    \thead{Nasal} &
    \na & \td{m} &  % bilabial
    \na & \na    &  % labiodental
    \na & \td{n} &  % dental
    \na & \na    &  % alveolar
    \na & \na    &  % palatal
    \na & \td{ŋ} &  % velar
    \na & \na    \\ % glottal
    \hline%---------------------------------------------------------------------
    \makecell{\rule{0pt}{1.5em}} &
    \thead{Frikativa} &
    \na    & \na    &  % bilabial
    \td{f} & \td{v} &  % labiodental
    \td{s} & \na    &  % dental
    \na    & \na    &  % alveolar
    \na    & \td{ʝ} &  % palatal
    \na    & \na    &  % velar
    \td{h} & \na    \\ % glottal
    \hline%---------------------------------------------------------------------
    \makecell{\rule{0pt}{1.5em}} &
    \thead{Approximant} &
    \na & \na    &  % bilabial
    \na & \na    &  % labiodental
    \na & \na    &  % dental
    \na & \td{ɹ} &  % alveolar
    \na & \na    &  % palatal
    \na & \na    &  % velar
    \na & \na    \\ % glottal
    \hline%---------------------------------------------------------------------
    \makecell{\rule{0pt}{1.5em}} &
    \thead{Lateral\\approximant} &
    \na & \na    &  % bilabial
    \na & \na    &  % labiodental
    \na & \td{l} &  % dental
    \na & \na    &  % alveolar
    \na & \na    &  % palatal
    \na & \na    &  % velar
    \na & \na    \\ % glottal
    \hline%---------------------------------------------------------------------
  \end{tabularx}

  \medskip

  ɧ {\normalsize \hspace{.25em} Tonlös dorso-palatal/velar frikativa}
  \hspace{.8em}
  ɕ {\normalsize \hspace{.25em} Tonlös alveolar-palatal frikativa}
}

%% [eof]
